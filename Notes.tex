\documentclass[11pt,a4paper]{article}

\usepackage[utf8]{inputenc}
\usepackage{parskip}
\usepackage{tabularx}
\usepackage{amsmath}
\usepackage{amssymb}
\usepackage{amsthm}
\usepackage{geometry}
\usepackage{booktabs}
\usepackage{centernot}
\usepackage{hyperref}
\usepackage{eufrak}
\usepackage{graphicx}
\graphicspath{{pics/}}
\geometry{a4paper, left=20mm, right=20mm, top=20mm, bottom=20mm}

\usepackage{fancyhdr}
\pagestyle{fancy}
\lhead{Anthony Catterwell}
\chead{\textsc{University of Edinburgh}}
\rhead{Analysis}

\title{Honours Analysis Notes}
\author{Anthony Catterwell}

\begin{document}
\maketitle
\tableofcontents

\break{}

\section{The Real Number System}

\subsection{Introduction}

\subsection{Ordered Field Axioms}

\begin{itemize}
    \item \textbf{Postulate 1} \emph{Field Axioms} \\
        There are functions $+$ and $\cdot$ defined on
        $\mathbb{R}^2 := \mathbb{R} \times \mathbb{R}$,
        which satisfy the following properties $\forall a, b, c \in \mathbb{R}$

        \begin{itemize}
            \item \emph{Closure Properties}:
                $a + b, \ a \cdot b \in \mathbb{R}$

            \item \emph{Associative Properties}:
                $a + (b + c) = (a + b) + c$ and
                $a \cdot (b \cdot c) = (a \cdot b) \cdot c$

            \item \emph{Commutative Properties}:
                $a + b = b + a$ and $a \cdot b = b \cdot a$

            \item \emph{Distributive Law}:
                $a \cdot (b + c) = a \cdot b + a \cdot c$

            \item \emph{Existence of Additive Identity}:
                There is a unique element $0 \in \mathbb{R}$ such that
                $0 + a = a$ for all $a \in \mathbb{R}$

            \item \emph{Existence of Multiplicative Identity}:
                There is a unique element $1 \in \mathbb{R}$ such that
                $1 \neq 0$ and $1 \cdot a = a$ for all $a \in \mathbb{R}$

            \item \emph{Existence of Additive Inverses}:
                For every $x \in \mathbb{R}$ there is a unique element $-x \in \mathbb{R}$ such that
                \[
                    x + (-x) = 0
                \]

            \item \emph{Existence of Multiplicative Inverses}:
                For every $x \in \mathbb{R} \setminus \{0\}$ there is a unique element
                $x^{-1} \in \mathbb{R}$ such that
                \[
                    x \cdot (x^{-1}) = 1
                \]
        \end{itemize}

    \item \textbf{Postulate 2} \emph{Order Axioms} \\
        There is a relation $<$ on $\mathbb{R} \times \mathbb{R}$ that has the following
        properties:
        \begin{itemize}
            \item \emph{Trichotomy Property}:
                Given $a, b \in \mathbb{R}$, one and only one of the following statements hold:
                \[
                    a < b, \ b < a, \ \text{or} \ a = b
                \]

            \item \emph{Transitive Property}:
                For $a, b, c, \in \mathbb{R}$
                \[
                    a < b \ \text{and} \b < c \implies a < c
                \]

            \item \emph{Additive Property}:
                For $a, b, c \in \mathbb{R}$
                \[
                    a < b \ \text{and} \ c \in \mathbb{R} \implies a + c < b + c
                \]

            \item \emph{Multiplicative Properties}:
                For $a, b, c, \in \mathbb{R}$
                \[
                    a < b \ \text{and} \ c > 0 \implies ac < bc
                \]
                and
                \[
                    a < b \ \text{and} \ c < 0 \implies bc < ac
                \]
        \end{itemize}

    \item \textbf{Remark 1.1} \\
        We will assume that the sets $\mathbb{N}$ and $\mathbb{Z}$ satisfy the following properties:
        \begin{enumerate}
            \item If $n, m \in \mathbb{Z}$, then $n + m, n - m$ and $mn$ belong to $\mathbb{Z}$
            \item If $n \in \mathbb{Z}$, then $n \in \mathbb{N}$ if and only if $n \geq 1$
            \item There is no $n \in \mathbb{Z}$ that satisfies $0 < n < 1$
        \end{enumerate}

    \item \textbf{Definition 1.4} \emph{Absolute Value} \\
        The \emph{absolute value} of a number $a \in \mathbb{R}$ is the number
        \[
            |a| :=
            \begin{cases}{}
                a  & a \geq 0 \\
                -a & a < 0 \\
            \end{cases}
        \]

    \item \textbf{Remark 1.5} The \emph{absolute value} is multiplicative;
        that is, $|ab| = |a||b| \ \forall a, b, \in \mathbb{R}$

    \item \textbf{Theorem 1.6} \emph{Fundamental Theorem of Absolute Values} \\
        Let $a \in \mathbb{R}$ and $M \geq 0$.
        Then $|a| \leq M \iff -M \leq a \leq M$.

    \item \textbf{Theorem 1.7} The absolute value satisfies the following three properties:
        \begin{enumerate}
            \item \emph{Positive Definite}:
                For all $a \in \mathbb{R}$, $|a| > 0$ with $|a| = 0$ if and only if $a = 0$.
            \item \emph{Symmetric}:
                For all $a, b, \in \mathbb{R}$, $|a - b| = |b - a|$,
            \item \emph{Triangle Inequalities}:
                For all $a, b \in \mathbb{R}$,
                \[
                    |a + b| \leq |a| + |b| \quad \text{and}
                    \quad \lvert |a| - |b| \rvert \leq |a - b|
                \]
        \end{enumerate}

    \item \textbf{Theorem 1.9} Let $x, y, a \in \mathbb{R}$
        \begin{enumerate}
            \item $x < y + \epsilon \ \forall \epsilon > 0 \iff x \leq y$
            \item $x > y - \epsilon \ \forall \epsilon > 0 \iff x \geq y$
            \item $|a| < \epsilon \ \forall \epsilon > 0 \iff a = 0$
        \end{enumerate}

\end{itemize}

\subsection{Completeness Axiom}

\begin{itemize}
    \item \textbf{Definition 1.10} \emph{Upper bounds} \\
        Let $E \subset \mathbb{R}$ be non-empty
        \begin{enumerate}
            \item The set $E$ is said to be \emph{bounded above} if and only if there is an
                $M \in \mathbb{R}$ such that $a \leq M$ for all $a \in E$,
                in which case $M$ is called an \emph{upper bound} of $E$.
            \item A number $s$ is called a \emph{supremum} of the set $E$ if and only if $s$
                is an upper bound of $E$ and $s \leq M$ for all upper bounds $M$ of $E$.
                (In this case we shall say that $E$ has a \emph{finite supremeum} $s$ and write
                $s = \sup E$)
        \end{enumerate}

    \item \textbf{Remark 1.12}
        If a set has one upper bound, it has infinitely many upper bounds.

    \item \textbf{Remark 1.13}
        If a set has a supremum, then it has only one supremum.

    \item \textbf{Theorem} \emph{Approximation Property for Suprema} \\
        If $E$ has a finite supremum and $\epsilon > 0$ is any positive number, then there is a
        point $a \in E$ such that
        \[
            \sup E - \epsilon < a \leq \sup E
        \]

    \item \textbf{Theorem 1.15} \\
        If $E \subset \mathbb{Z}$ has a supremum, then $\sup E \in E$.
        In particular, if the supremum of a set, which contains only integers, exists,
        that supremum must be an integer.

    \item \textbf{Postulate 3} \emph{Completeness Axiom} \\
        If $E$ is a nonempty subset of $\mathbb{R}$ that is bounded above, then $E$ has a finite
        supremum.

    \item \textbf{Theorem 1.16} \emph{The Archimedean Principle} \\
        Given real numbers $a$ and $b$, with $a > 0$, there is an integer $n \in \mathbb{N}$
        such that $b < na$.

    \item \textbf{Theorem 1.18} \emph{Density of Rationals} \\
        If $a, b, \in \mathbb{R}$ satisfy $a < b$, then there is a $q \in \mathbb{Q}$
        such that $a < q < b$.

    \item \textbf{Definition 1.19} \emph{Upper bounds} \\
        Let $E \in \mathbb{R}$ be nonempty
        \begin{enumerate}
            \item The set $E$ is said to be \emph{bounded below} if and only if there is an
                $m \in \mathbb{R}$ such that $a \geq E$, in which case $m$ is called a
                \emph{lower bound} of the set $E$.
            \item A number $t$ is called an \emph{infimum} of the set $E$ if and only if $t$
                is a lower bound of $E$ and $t \geq m$ and write $t = \inf E$.
            \item $E$ is said to be \emph{bounded} if and only if it is bounded both above and
                below.
        \end{enumerate}

    \item \textbf{Theorem 1.20} \emph{Reflection Principle} \\
        Let $E \in \mathbb{R}$ be nonempty
        \begin{enumerate}
            \item $E$ has a supremum if and only if $-E$ has an infimum, in which case
                \[
                    \inf (-E) = -\sup E
                \]
            \item $E$ has an infimum if and only if $-E$ has a supremum, in which case
                \[
                    \sup (-E) = -\inf E
                \]
        \end{enumerate}

    \item \textbf{Theorem 1.21} \emph{Monotone Property} \\
        Suppose that $A \subseteq B$ are nonempty subsets of $\mathbb{R}$.
        \begin{enumerate}
            \item If $B$ has a supremum, then $\sup A \leq \sup B$.
            \item If $B$ has an infimum, then $\inf A \geq \inf B$.
        \end{enumerate}

\end{itemize}

\subsection{Mathematical Induction}

\begin{itemize}
    \item \textbf{Theorem 1.22} \emph{Well-Ordering Principle} \\
        If $E$ is a nonempty subset of $\mathbb{N}$, then $E$ has a least element
        (i.e.\ $E$ has a finite infimum and $\inf E \in E$).

    \item \textbf{Theorem 1.23} \\
        Suppose for each $n \in \mathbb{N}$ that $A(n)$ is a proposition which satisfies
        the following two properties:
        \begin{enumerate}
            \item $A(1)$ is true.
            \item For every $n \in \mathbb{N}$ for which $A(n)$ is true, $A(n+1)$ is also true.
        \end{enumerate}
        Then $A(n)$ is true for all $n \in \mathbb{N}$.

    \item \textbf{Theorem 1.26} \emph{Binomial Formula} \\
        If $a, b, \in \mathbb{R}, n \in \mathbb{N}$ and $0^0$ is interpreted to be $1$, then
        \[
            {(a+b)}^n = \sum_{k=0}^n
            \begin{pmatrix}
                n \\ k
            \end{pmatrix}
            a^{n-k} b^k
        \]

\end{itemize}

\subsection{Inverse Functions and Images}

\begin{itemize}
    \item \textbf{Definition 1.29} \emph{Injection, Surjection, Bijection} \\
        Let $X$ and $Y$ be sets and $f : X \to Y$
        \begin{enumerate}
            \item $f$ is said to be \emph{injective} if and only if
                \[
                    x_1, x_2 \in X \ \text{and} \ f(x_1) = f(x_2) \implies x_1 = x_2
                \]
            \item $f$ is said to be \emph{surjective} if and only if
                \[
                    \forall y \in Y \ \exists x \in X \backepsilon y = f(x)
                \]

            \item $f$ is called \emph{bijective} if and only if it is both injective
                and surjective
        \end{enumerate}

    \item \textbf{Theorem 1.30} \\
        Let $X$ and $Y$ be sets and $f : X \to Y$.
        Then the following three statements are equivalent.
        \begin{enumerate}
            \item $f$ has an inverse;
            \item $f$ is injective from $X$ onto $Y$;
            \item There is a function $g : Y \to X$ such that
                \[
                    g(f(x)) = x \quad \forall x \in X
                \]
                and
                \[
                    f(g(y)) = y \quad \forall y \in Y
                \]
        \end{enumerate}
        Moreover, for each $f : X \to Y$, there is only one function $g$ that satisfies these.
        It is the inverse function $f^{-1}$.

    \item \textbf{Remark 1.31} \\
        Let $I$ be an interval and let $f : I \to \mathbb{R}$.
        If the derivative of $f$ is either always positive on $I$, or always negative on $I$,
        then $f$ is injective on $I$.

    \item \textbf{Definition 1.33} \emph{Image} \\
        Let $X$ and $Y$ be sets and $f : X \to Y$.
        The \emph{image} of a set $E \subseteq X$ under $f$ is the set
        \[
            f(E) := \{ y \in Y : y = f(x) \ \text{for some} \ x \in E \}
        \]
        The \emph{inverse image} of a set $E \subseteq Y$ under $f$ is the set
        \[
            f^{-1}(E) := \{ x \in X : f(x) = y \ \text{for some} \ y \in E \}
        \]

    \item \textbf{Definition 1.35} \emph{Union, Intersection} \\
        Let $\mathcal{E} = {\{ E_{\alpha} \}}_{\alpha \in A}$ be a collection of sets.
        \begin{enumerate}
            \item The \emph{union} of the collection $\mathcal{E}$ is the set
                \[
                    \bigcup_{\alpha \in A} E_\alpha := \{ x : x \in E_\alpha \ \text{for some} \
                    \alpha \in A \}
                \]
            \item The \emph{intersection} of the collection $\mathcal{E}$ is the set
                \[
                    \bigcap_{\alpha \in A} E_\alpha := \{ x : x \in E_\alpha \ \text{for all} \
                    \alpha \in A \}.
                \]
        \end{enumerate}

    \item \textbf{Theorem 1.36} \emph{DeMorgan's Laws} \\
        Let $X$ be a set and ${\{ E_\alpha \}}_{\alpha \in A}$ be a collection of subsets of $X$.
        If for each $E \subseteq X$ the symbol $E^c$ represents the set $X \setminus E$, then
        \[
            {\left( \bigcup_{\alpha \in A} E_\alpha \right)}^c =
            \bigcap_{\alpha \in A} E_\alpha^c
        \]
        and
        \[
            {\left( \bigcap_{\alpha \in A} E_\alpha \right)}^c =
            \bigcup_{\alpha \in A} E_\alpha ^ c
        \]

    \item \textbf{Theorem 1.37} \\
        Let $X$ and $Y$ be sets and $f : X \to Y$.
        \begin{enumerate}
            \item If ${\{ E_\alpha \}}{\alpha \in A}$ is a collection of subsets of $X$, then
                \[
                    f \left( \bigcup_{\alpha \in A} E_\alpha \right) =
                    \bigcup_{\alpha \in A} f(E_\alpha)
                    \quad \text{and} \quad
                    f \left( \bigcap_{\alpha \in A} E_\alpha \right) \subseteq
                    \bigcap_{\alpha \in A} f(E_\alpha)
                \]
            \item If $B$ and $C$ are subsets of $X$, then
                $f(C) \setminus f(B) \subseteq f(C \setminus B)$
            \item If ${\{E_\alpha\}}_{\alpha \in A}$ is a collection of subsets of $Y$, then
                \[
                    f^{-1} \left( \bigcup_{\alpha \in A} E_\alpha \right) =
                    \bigcup_{\alpha \in A} f^{-1} (E_\alpha)
                    \quad \text{and} \quad
                    f^{-1} \left( \bigcap_{\alpha \in A} E_\alpha \right) =
                    \bigcap_{\alpha \in A} f^{-1} (E_\alpha)
                \]
            \item If $B$ and $C$ are subsets of $Y$, then
                $f^{-1}(C \setminus B) = f^{-1}(C) \setminus f^{-1}(B)$.
            \item If $E \subseteq f(x)$, then $f(f^{-1}(E)) = E$, but if $E \subseteq X$,
                then $E \subseteq f^{-1}(f(E))$.
        \end{enumerate}

\end{itemize}

\subsection{Countable and Uncountable Sets}

\begin{itemize}
    \item \textbf{Definition 1.38} \emph{Countable & Uncountable}
        Let $E$ be a set.
        \begin{enumerate}
            \item $E$ is said to be \emph{finite} if and only if either $E = \emptyset$
                or there exists an injective function which takes
                $\{ 1, 2, \ldots, n \}$ onto $E$, for some $n \in \mathbb{N}$.
            \item $E$ is said to be \emph{countable} if and only if there exists and injective
                function which takes $\mathbb{N}$ onto $E$.
            \item $E$ is said to be \emph{at most countable} if and only if $E$ is either
                finite or countable.
            \item $E$ is said to be \emph{uncountable} if and only if $E$ is neither finite nor
                countable.
        \end{enumerate}

    \item \textbf{Remark 1.39} \emph{Cantor's Diagonalisation Argument} \\
        The open interval $(0, 1)$ is uncountable.

    \item \textbf{Lemma 1.40} \\
        A nonempty set $E$ is at most countable if and only if there is a function $g$ from
        $\mathbb{N}$ onto $E$.

    \item \textbf{Theorem 1.41} \\
        Suppose $A$ and $B$ are sets.
        \begin{enumerate}
            \item If $A \subseteq B$ and $B$ is at most countable, then $A$ is at most countable.
            \item If $A \subseteq B$ and $A$ is uncountable, then $B$ is uncountable.
            \item $\mathbb{R}$ is uncountable.
        \end{enumerate}

    \item \textbf{Theorem 1.42} \\
        Let $A_1, A_2, \ldots$ be at most countable sets.
        \begin{enumerate}
            \item Then $A_1 \times A_2$ is at most countable.
            \item If
                \[
                    E = \bigcup_{j=1}^\infty A_j :=
                    \bigcup_{j \in \mathbb{N}} A_j :=
                    \{ x : x \in A_j \ \text{for some} \ j \in \mathbb{N} \},
                \]
                then $E$ is at most countable.
        \end{enumerate}

    \item \textbf{Remark 1.43} \\
        The sets $\mathbb{Z}$ and $\mathbb{Q}$ are countable, but the set of irrationals is
        uncountable.

\end{itemize}

\break{}

\section{Sequences in $\mathbb{R}$}

\subsection{Limits of Sequences}

\begin{itemize}
    \item \textbf{Definition 2.1} \emph{Convergence} \\
        A sequence of real numbers $\{ x_n \}$ is set to \emph{converge} to a real number
        $a \in \mathbb{R}$ if and only if for every $\epsilon > 0$ there is an $N \in \mathbb{N}$
        (which in general depends on $\epsilon$) such that
        \[
            n \geq N \implies |x_n - a| < \epsilon
        \]

    \item \textbf{Remark 2.4}
        A sequence can have at most one limit.

    \item \textbf{Definition 2.5} \emph{Subsequence} \\
        By a \emph{subsequence} of a sequence ${\{x_n\}}_{n \in \mathbb{N}}$,
        we shall mean a sequence of the form ${\{x_n_k\}}_{k \in \mathbb{N}}$,
        where each $n_k \in \mathbb{N}$ and $n_1 < n_2 < \cdots$.

    \item \textbf{Remark 2.6} \\
        If ${ \{x_n\} }_{n \in \mathbb{N}}$ converges to $a$ and
        ${ \{x_n_k\} }{k \in \mathbb{N}}$ is any subsequence of
        ${ \{x_n\} }_{n \in \mathbb{N}}$, then $x_n_k$ converges to $a$ as $k \to \infty$.

    \item \textbf{Definition 2.7} \emph{Bounded Sequences} \\
        Let $\{ x_n \}$ be a sequence of real numbers.
        \begin{enumerate}
            \item The sequence $\{ x_n \}$ is said to be \emph{bounded above} if and only if
                the set $ \{ x_n : n \in \mathbb{N} \}$ is bounded above.
            \item The sequence $\{ x_n \}$ is said to be \emph{bounded below} if and only if
                the set $\{ x_n : n \in \mathbb{N} \}$ is bounded below.
            \item $\{ x_n \}$ is said to be \emph{bounded} if and only if it is bounded both
                above and below.
        \end{enumerate}

    \item \textbf{Theorem 2.8}
        Every convergent sequence is bounded.
\end{itemize}

\subsection{Limit Theorems}

\begin{itemize}
    \item \textbf{Theorem 2.9} \emph{Squeeze Theorem} \\
        Suppose that $\{x_n\}, \{y_n\}$, and $\{w_n\}$ are real sequences.
        \begin{enumerate}
            \item If $x_n \to a$ and $y_n \to a$ as $n \to \infty$,
                and if there is an $N_0 \in \mathbb{N}$ such that
                \[
                    x_n \leq w_n \leq y_n \ \text{for} \ n \geq N_0
                \]
                then $w_n \to a$ as $n \to \infty$.
            \item If $x_n \to 0$ as $n \to \infty$ and $\{y_n\}$ is bounded,
                then $x_n y_n \to 0$ as $n \to \infty$.
        \end{enumerate}

    \item \textbf{Theorem 2.11} \\
        Let $E \subset \mathbb{R}$.
        If $E$ has a finite supremum (respectively, a finite infimum),
        then there is a sequence $x_n \in E$ such that $x_n \to \sup E$
        (respectively, a sequence $y_n \in E$ such that $y_n \to \inf E$)
        as $n \to \infty$.

    \item \textbf{Theorem 2.12} \\
        Suppose that $\{x_n\}$ and $\{y_n\}$ are real sequences and that $\alpha \in \mathbb{R}$.
        If $\{x_n\}$ and $\{y_n\}$ are convergent, then
        \begin{enumerate}
            \item
                \[
                    \lim_{n \to \infty} (x_n + y_n) =
                    \lim_{n \to \infty} x_n + \lim_{n \to \infty} y_n
                \]
            \item
                \[
                    \lim_{n \to \infty} (\alpha x_n) = \alpha \lim_{n \to \infty} x_n
                \]
                and
            \item
                \[
                    \lim_{n \to \infty} (x_n y_n) =
                    ( \lim_{n \to \infty} x_n ) (\lim_{n \to \infty} y_n)
                \]
                If, in addition, $y_n \neq 0$ and $\lim_{n \to \infty} y_n \neq 0$, then
            \item
                \[
                    \lim_{n \to \infty} \frac{x_n}{y_n} =
                    \frac{\lim_{n \to \infty} x_n}{\lim_{n \to \infty} y_n}
                \]
                (In particular, all these limits exist.)

        \end{enumerate}

    \item \textbf{Definition 2.14} \emph{Divergence} \\
        Let $\{x_n\}$ be a sequence of real numbers.
        \begin{enumerate}
            \item $\{x_n\}$ is said to \emph{diverge} to $+\infty$ if and only if for each
                $M \in \mathbb{R}$ there is an $N \in \mathbb{N}$ such that
                \[
                    n \geq N \implies x_n > M
                \]
            \item $\{x_n\}$ is said to \emph{diverge} to $-\infty$ if and only if for each
                $M \in \mathbb{R}$ there is an $N \in \mathbb{N}$ such that
                \[
                    n \geq N \implies x_n < M
                \]
        \end{enumerate}

    \item \textbf{Theorem 2.15} \\
        Suppose that $\{x_n\}$ and $\{y_n\}$ are real sequences such that $x_n \to +\infty$
        (respectively, $x_n \to -\infty$) as $n \to \infty$.
        \begin{enumerate}
            \item If $y_n$ is bounded below (respectively, $y_n$ is bounded above), then
                \[
                    \lim_{n \to \infty} (x_n + y_n) = +\infty \quad
                    (\text{respectively}, \ \lim_{n \to \infty} (x_n + y_n) = -\infty)
                \]

            \item If $\alpha > 0$, then
                \[
                    \lim_{n \to \infty} (\alpha x_n) = +\infty \quad
                    (\text{respectively}, \ \lim_{n \to \infty} (\alpha x_n) = -\infty)
                \]

            \item If $y_n > M_0$ for some $M_0 > 0$ and all $n \in \mathbb{N}$, then
                \[
                    \lim_{n \to \infty} (x_n y_n) = +\infty \quad
                    (\text{respectively}, \ \lim_{n \to \infty} (x_n y_n) = -\infty)
                \]

            \item If $\{y_n\}$ is bounded and $x_n \neq 0$, then
                \[
                    \lim_{n \to \infty} \frac{y_n}{x_n} = 0
                \]

        \end{enumerate}

    \item \textbf{Corollary 2.16} \\
        Let $\{x_n\}$, $\{y_n\}$ be real sequences and $\alpha, x, y$ be extended real numbers.
        If $x_n \to x$ and $y_n \to y$, as $n \to \infty$, then
        \[
            \lim_{n \to \infty} (x_n + y_n) = x + y
        \]
        provided that the right side is not of the form $\infty - \infty$, and
        \[
            \lim_{n \to \infty} (\alpha x_n) = \alpha x, \quad \lim_{n \to \infty} (x_n y_n) = xy
        \]
        provided that none of these products is of the form $0 \cdot \pm \infty$.

    \item \textbf{Theorem 2.17} \emph{Comparison Theorem} \\
        Suppose that $\{x_n\}$ and $\{y_n\}$ are convergent sequences.
        If there is an $N_0 \in \mathbb{N}$ such that
        \[
            x_n \leq y_n \ \text{for} \ n \geq N_0
        \]
        then
        \[
            \lim_{n \to \infty} x_n \leq \lim_{n \to \infty} y_n
        \]
        In particular, if $x_n \in [a, b]$ converges to some point $c$,
        then $c$ must belong to $[a, b]$.

\end{itemize}

\subsection{Bolzano-Weierstrass Theorem}

\begin{itemize}
    \item \textbf{Definition 2.18} \emph{Increasing, Decreasing}
        Let ${\{x_n\}}_{n \in \mathbb{N}}$ be a sequence of real numbers.
        \begin{enumerate}
            \item $\{x_n\}$ is said to be \emph{increasing}
                (respectively, \emph{strictly increasing}) if and only if
                $x_1 \leq x_2 \leq \cdots (\text{respectively}, x_1 < x_2 < \cdots)$.
            \item $\{x_n\}$ is said to be \emph{decreasing}
                (respectively, \emph{strictly decreasing}) if and only if
                $x_1 \geq x_2 \geq \cdots (\text{respectively}, x_1 > x_2 > \cdots)$.
            \item $\{x_n\}$ is said to be \emph{monotone} if and only if it is either
                increasing or decreasing.
        \end{enumerate}

    \item \textbf{Theorem 2.19} \emph{Monotone Convergence Theorem} \\
        if $\{x_n\}$ is increasing and bounded above, or if $\{x_n\}$ is decreasing and bounded
        below, then $\{x_n\}$ converges to a finite limit.

    \item \textbf{Definition 2.22} \emph{Nested} \\
        A sequence of sets ${\{I_n\}}_{n \in \mathbb{N}}$ is said to be \emph{nested}
        if and only if
        \[
            I_1 \supseteq I_2 \supseteq \cdots
        \]


    \item \textbf{Theorem 2.23} \emph{Nested Interval Property} \\
        If ${\{I_n\}}_{n \in \mathbb{N}}$ is a nested sequence of nonempty closed bounded
        intervals, then $E := \bigcap_{n=1}^\infty I_n$ is nonempty.
        Moreover, if the lengths of these intervals satisfy $|I_n \to 0$ as $n \to \infty$
        then $E$ is a single point.

    \item \textbf{Remark 2.24}
        The Nested Interval Property might not hold if ``closed'' is omitted.

    \item \textbf{Remark 2.25}
        The Nested Interval Property might not hold if ``bounded'' is omitted.

    \item \textbf{Theorem 2.26} \emph{Bolzano-Weierstrass Theorem} \\
        Every bounded sequence of real numbers has a convergent subsequence.

\end{itemize}

\subsection{Cauchy Sequences}

\begin{itemize}
    \item \textbf{Definition 2.27} \emph{Cauchy} \\
        A sequence of points $x_n \in \mathbb{R}$ is said to be \emph{Cauchy} (in \mathbb{R})
        if and only if for every $\epsilon > 0$ there is an $N \in \mathbb{N}$ such that
        \[
            n, m \geq N \implies |x_n - x_m| < \epsilon
        \]

    \item \textbf{Remark 2.28}
        If $\{x_n\}$ is convergent, then $\{x_n\}$ is Cauchy.

    \item \textbf{Theorem 2.29} \emph{Cauchy} \\
        Let $\{x_n\}$ be a sequence of real numbers.
        Then $\{x_n\}$ is Cauchy if and only if $\{x_n\}$ converges
        (to some point $a \in \mathbb{R}$).

    \item \textbf{Remark 2.31}
        A sequence that satisfies $x_{n+1} - x_n \to 0$ is not necessarily Cauchy.

\end{itemize}

\subsection{Limits Supremum and Infimum}

\begin{itemize}
    \item \textbf{Definition 2.32} \emph{Limit Supremum \& Infimum} \\
        Let $\{x_n\}$ be a real sequence.
        Then the \emph{limit supremum} of $\{x_n\}$ is the extended real number
        \[
            \limsup_{n \to \infty} x_n := \lim_{n \to \infty} (\sup_{k \geq n} x_k)
        \]
        and the \emph{limit infimum} of $\{x_n\}$ is the extended real number
        \[
            \liminf_{n \to \infty} x_n := \lim_{n \to \infty} (\inf_{k \geq n} x_k)
        \]

    \item \textbf{Theorem 2.35} \\
        Let $\{x_n\}$ be a sequence of real numbers, $s = \limsup_{n \to \infty} x_n$,
        and $t = \liminf_{n \to \infty} x_n$.
        Then there are subsequences ${\{x_n_k\}}_{k \in \mathbb{N}}$ and
        ${\{x_\ell_j\}}_{j \in \mathbb{N}}$ such that
        $x_n_k \to s$ as $k \to \infty$ and $x_\ell_j \to t$ as $j \to \infty$.

    \item \textbf{Theorem 2.36} \\
        Let $\{x_n\}$ be a real sequence and $x$ be an extended real number.
        Then $x_n \to x$ as $n \to \infty$ if and only if
        \[
            \limsup_{n \to \infty} x_n = \liminf_{n \to \infty} x_n = x
        \]

    \item \textbf{Theorem 2.37} \\
        Let $\{x_n\}$ be a sequence of real numbers.
        Then $\limsup_{n \to \infty} x_n$ (respectively, $\liminf_{n \to \infty}$) is the largest
        value (respectively, the smallest value) to which some subsequences of $\{x_n\}$
        converges.
        Namely, if $x_n_k \to x$ as $k \to \infty$, then
        \[
            \liminf_{n \to \infty} x_n \leq x \leq \limsup_{n \to \infty} x_n
        \]

    \item \textbf{Remark 2.38}
        If $\{x_n\}$ is any sequence of real numbers, then
        \[
            \liminf_{n \to \infty} x_n \leq \limsup_{n \to \infty} x_n
        \]

    \item \textbf{Remark 2.39}
        A real sequence $\{x_n\}$ is bounded above if and only if
        $\limsup_{n \to \infty} x_n < \infty$, and is bounded below if and only if
        $\liminf_{n \to \infty} x_n > -\infty$.

    \item \textbf{Theorem 2.40} \\
        If $x_n \leq y_n$ for $n$ large, then
        \[
            \limsup_{n \to \infty} x_n \leq \limsup_{n \to \infty} y_n \quad \text{and} \quad
            \liminf_{n \to \infty} y_n \leq \liminf_{n \to \infty} y_n
        \]

\end{itemize}

\break{}

\section{Functions on \mathbb{R}}

\subsection{Two-Sided Limits}

\begin{itemize}
    \item \textbf{Definition 3.1} \emph{Limits} \\
        Let $a \in \mathbb{R}$, let $I$ be an open interval which contains $a$, and let $f$ be
        a real function defined everywhere on $I$ except possibly at $a$.
        Then $f(x)$ is said to \emph{converge to $L$, as $x$ approaches $a$},
        if and only if for every $\epsilon > 0$ there is a $\delta > 0$
        (which in general depends on $\epsilon, f, I$, and $a$) such that
        \[
            0 < |x - a| < \delta \implies |f(x) - L| < \epsilon
        \]
        In this case we write
        \[
            L = \lim_{x \to a} f(x) \quad \text{or} \quad f(x) \to L \ \text{as} \ x \to a
        \]
        and call $L$ the \emph{limit} of $f(x)$ as $x$ approaches $a$.

    \item \textbf{Remark 3.4} \\
        Let $a \in \mathbb{R}$, let $I$ be an open interval which contains $a$, and let
        $f$, $g$ be real functions defined everywhere on $I$ except possibly at $a$.
        If $f(x) = g(x)$ for all $x \in I \setminus \{a\}$ and $f(x) \to L$ as
        $x \to a$, then $g(x)$ also has a limit as $x \to a$, and
        \[
            \lim_{x \to a} g(x) = \lim_{x \to a} f(x)
        \]

    \item \textbf{Theorem 3.6} \emph{Sequential Characterisation of Limits} \\
        Let $a \in \mathbb{R}$, let $I$ be an open interval which contains $a$, and let
        $f$ be a real function defined everywhere on $I$ except possibly at $a$.
        Then
        \[
            L = \lim_{x \to a} f(x)
        \]
        exists if and only if $f(x_n) \to L$ as $n \to \infty$ for every sequence
        $\{x_n\} \in I \setminus \{a\}$ which converges to $a$ as $n \to \infty$.

    \item \textbf{Theorem 3.8} \\
        Suppose that $a \in \mathbb{R}$, that $I$ is an open interval which contains $a$,
        and that $f$, $g$, are real functions defined everywhere on $I$ except possibly
        at $a$.
        If $f(x)$ and $g(x)$ converge as $x$ approaches $a$, then so do
        $(f + g)(x), (fg)(x), (\alpha f)(x)$, and $(f/g)(x)$
        (when the limit of $g(x)$ is nonzer).
        In fact,
        \begin{align*}{}
            \lim_{x \to a} (f + g)(x) &= \lim_{x \to a} + \lim_{x \to a} g(x) \\
            \lim_{x \to a} (\alpha f)(x) &= \aplha \lim_{x \to a} f(x) \\
            \lim_{x \to a} (fg)(x) &= \lim_{x \to a} \lim_{x \to a} g(x) \\
        \end{align*}
        and (when the limit of $g(x)$ is nonzero)
        \[
            \lim_{x \to a} \left( \frac{f}{g} \right) (x) &=
            \frac{\lim_{x \to a} f(x)}{\lim_{x \to a} g(x)}
        \]

    \item \textbf{Theorem 3.9} \emph{Squeeze Theorem for Functions} \\
        Suppose that $a \in \mathbb{R}$, that $I$ is an open interval which contains $a$,
        and that $f, g, h$ are real functions defined everywhere on $I$ except possibly at $a$.
        \begin{enumerate}
            \item If $g(x) \leq h(x) \leq f(x) \ \forall x \in I \setminus \{a\}$, and
                \[
                    \lim_{x \to a} f(x) = \lim_{x \to a} g(x) = L,
                \]
                then the limit of $h(x)$ exists, as $x \to a$, and
                \[
                    \lim_{x \to a} h(x) = L.
                \]
            \item If $|g(x)| \leq M \ \forall x \in I \setminus \{a\}$ and $f(x) \to 0$
                as $x \to a$, then
                \[
                    \lim_{x \to a} f(x) g(x) = 0
                \]

        \end{enumerate}

    \item \textbf{Theorem 3.10} \emph{Comparison Theorem for Functions} \\
        Suppose that $a \in \mathbb{R}$, that $I$ is an open interval which contains $a$,
        and that $f, g$ are real functions defined everywhere on $I$ except possibly at $a$.
        If $f$ and $g$ have a limit as $x$ approaches $a$ and
        $f(x) \leq g(x) \ \forall x \in I \setminus \{a\}$, then
        \[
            \lim_{x \to a} f(x) \leq \lim_{x \to a} g(x)
        \]

\end{itemize}

\subsection{One-Sided Limits and Limits at Infinity}

\begin{itemize}
    \item \textbf{Definition 3.12} \emph{Converge from left \& right} \\
        Let $a \in \mathbb{R}$ and $f$ be a real function.
        \begin{enumerate}
            \item $f(x)$ is said to \emph{converge to $L$ as $x$ approaches $a$ from the right}
                if and only if $f$ is defined on some open interval $I$ with left endpoint $a$
                and for every $\epsilon > 0$ there is a $\delta > 0$
                (which in general depends on $\epsilon, f, I$, and $a$) such that
                \[
                    a + \delta \in I \quad \text{and} \quad a < x < a + \delta \implies
                    |f(x) - L| < \epsilon
                \]
                in this case we call $L$ the \emph{right-hand limit} of $f$ at $a$,
                and denote it by
                \[
                    f(a+) := L =: \lim_{x \to a+} f(x)
                \]
            \item $f(x)$ is said to \emph{converge to $L$ as $x$ approaches $a$ from the left}
                if and only if $f$ is defined on some open interval $I$ with left endpoint $a$
                and for every $\epsilon > 0$ there is a $\delta > 0$
                (which in general depends on $\epsilon, f, I$, and $a$) such that
                \[
                    a + \delta \in I \quad \text{and} \quad a < x < a + \delta \implies
                    |f(x) - L| < \epsilon
                \]
                in this case we call $L$ the \emph{left-hand limit} of $f$ at $a$,
                and denote it by
                \[
                    f(a-) := L =: \lim_{x \to a-} f(x)
                \]
        \end{enumerate}

    \item \textbf{Theorem 3.14} \\
        Let $f$ be a real function.
        Then the limit
        \[
            \lim_{x \to a} f(x)
        \]
        exists and equals $L$ if and only if
        \[
            L = \lim_{x \to a+} f(x) = \lim_{x \to a-} f(x)
        \]

    \item \textbf{Definition 3.15} \emph{Convergence} \\
        Let $a, L, \in \mathbb{R}$ and let $f$ be a real function.
        \begin{enumerate}
            \item $f(x)$ is said to \emph{converge} to $L$ as $x \to \infty$ if and only if
                there exists a $c > 0$ such that $(c, \infty) \subset \mathrm{Dom}(f)$
                and given $\epsilon > 0$ there is an $M \in \mathbb{R}$ such that
                $x > M$ implies $|f(x) - L| < \epsilon$, in which case we shall write
                \[
                    \lim_{x \to \infty} f(x) = L \quad \text{or} \quad
                    f(x) \to L \ \text{as} \ x \to \infty
                \]
                Similarly, $f(x)$ is said to \emph{converge} to $L$ as $x \to -\infty$
                if and only if there exists a $c > 0$ such that
                $(\-infty, -c) \subset \mathrm{Dom}(f)$ and given $\epsilon > 0$ there is
                $M \in \mathbb{R}$ such that $x > M$ implies $|f(x) - L| < \epsilon$,
                in which case we shall write
                \[
                    \lim_{x \to \infty} = L \quad \text{or} \quad
                    f(x) \to L \ \text{as} \ x \to \infty
                \]
            \item The function $f(x)$ is said to converge to $\infty$ as $x \to a$
                if and only if there is an open interval $I$ containing $a$ such that
                $I \setminus \{a\} \subset \mathrm{Dom}(f)$ and given $M \in \mathrm{R}$
                there is a $\delta > 0$ such that $0 \leq |x - a| < \delta$ implies
                $f(x) < M$, in which case we shall write
                \[
                    \lim_{x \to a} f(x) = \infty \quad \text{or} \quad
                    f(x) \to \infty \ \text{as} \ x \to a
                \]
                Similarly, $f(x)$ is said to \emph{converge} to $-\infty$ as $x \to a$
                if and only if there is an open interval $I$ containing $a$ such that
                $I \setminus \{a\} \subset \mathrm{Dom}(f)$ and given $M \in \mathbb{R}$
                there is a $\delta > 0$ such that $0 < |x-a| < \delta$ implies $f(x) < M$,
                in which case we shall write
                \[
                    \lim_{x \to a} f(x) = -\infty \quad \text{or} \quad
                    f(x) \to -\infty \ \text{as} \ x \to a
                \]

        \end{enumerate}

    \item \textbf{Theorem 3.17} \\
        Let $a$ be an extended real number, and let $I$ be a nondegenerate open interval which
        either contains $a$ or has $a$ as one of its endpoints.
        Suppose further that $f$ is a real function defined on $I$ except possibly at $a$.
        Then
        \[
            \lim_{x \to a; x \in I} f(x)
        \]
        exists and equals $L$ if and only if $f(x_n) \to L$ for all sequences $x_n \in I$
        which satisfy $x_n \neq a$ and $x_n \to a$ as $n \to \infty$.
\end{itemize}

\subsection{Continuity}

\begin{itemize}
    \item \textbf{Definition 3.19} \emph{Continuous} \\
        Let $E$ be a nonempty subset of $\mathbb{R}$ and $f : E \to \mathbb{R}$.
        \begin{enumerate}
            \item $f$ is said to be \emph{continuous at a point $a \in \mathbb{E}$} if and only
                if given $\epsilon > 0$ there is a $\delta > 0$
                (which in general depends on $\epsilon, f$, and $a$) such that
                \[
                    |x-a| < delta \quad \text{and} \quad x \in E \implies
                    |f(x) - f(a)| < \epsilon
                \]

            \item $f$ is said to be \emph{continuous on $E$} if and only if $f$ is continuous
                at every $x \in E$.
        \end{enumerate}
    \item \textbf{Remark 3.20} \\
        Let $I$ be an open interval which contains a point $a$ and $f : I \to \mathbb{R}$.
        Then $f$ is continuous at $a \in \mathbb{I}$ if and only if
        \[
            f(a) = \lim_{x \to a} f(x)
        \]

    \item \textbf{Theorem 3.21} \\
        Suppose that $E$ is a nonempty subset of $\mathbb{R}$, that $a \in E$, and that
        $f : E \to \mathbb{R}$.
        Then the following statements are equivalent:
        \begin{enumerate}
            \item $f$ is continuous at $a \in E$.
            \item If $x_n$ converges to $a$ and $x_n \in E$, then
                $f(x_n) \to f(a)$ as $n \to \infty$.
        \end{enumerate}

    \item \textbf{Theorem 3.22} \\
        Let $E$ be a nonempty subset of $\mathbb{R}$ and $f, g : E \to \mathbb{R}$.
        If $f, g$ are continuous at a point $a \in E$ (respectively continuous on the set $E$),
        then so are $f+g$, $fg$, and $\alpha f$ (for any $\alpha \in \mathbb{R}$).
        Moreover, $f/g$ is continuous at $a \in E$ when $g(a) \neq 0$
        (respectively, on $E$ when $g(x) \neq 0\ \forall x \in E$).

    \item \textbf{Definition 3.23} \emph{Composition} \\
        Suppose that $A$ and $B$ are subsets of \mathbb{R}, that $f : A \to \mathbb{R}$
        and $g : B \to \mathbb{R}$.
        If $F(A) \subseteq B$ for every $x \in A$, then the \emph{composition} of $g$ with $f$
        is the function $g \circ f : A \to \mathbb{R}$ defined by
        \[
            (g \circ f) (x) := g(f(x)), \quad x \in A
        \]

    \item \textbf{Theorem 3.24} \\
        Suppose that $A$ and $B$ are subsets of $\mathbb{R}$, that $f : A \to \mathbb{R}$
        and $g : B \to \mathbb{R}$, and that $f(x) \in B \ \forall x \in A$.
        \begin{enumerate}
            \item If $A := I \setminus \{a\}$, where $I$ is a nondegenerate interval which
                either contains $a$ or has $a$ as one of its endpoints, if
                \[
                    L := \lim_{x \to a; x \in I} f(x)
                \]
                exists and belongs to $B$, and if $g$ is continuous and $L \in B$, then
                \[
                    (g \circ f) (x) =
                    g \left( \lim_{x \to a; x \in I} f(x)\right)
                \]
            \item If $f$ is continuous at $a \in A$ and $g$ is continuous at $f(a) \in B$,
                then $g \circ f$ is continuous at $a \in A$.
        \end{enumerate}

    \item \textbf{Definition 3.25} \emph{Bounded} \\
        Let $E$ be a nonempty subset of $\mathbb{R}$.
        A function $f : E \to \mathbb{R}$ is said to be \emph{bounded} on E if and only if
        there is an $M \in \mathbb{R}$ such that $|f(x)| \leq M$ for all $x \in E$, in
        which case we shall say that $f$ is \emph{dominated} by $M$ on $E$.

    \item \textbf{Theorem 3.26} \emph{Extreme Value Theorem} \\
        If $I$ a is closed, bounded interval and $f : I \to \mathbb{R}$ is continuous on $I$,
        then $f$ is bounded on $I$.
        Moreover if
        \[
            M = \sup_{x \in I} f(x) \quad \text{and} \quad m = \inf_{x \in I} f(x)
        \]
        then there exist points $x_m, x_M \in I$ such that
        \[
            f(x_M) = M \quad \text{and} \quad f(x_m) = m
        \]


    \item \textbf{Remark 3.27}
        The Existence Value Theorem is false if either ``closed'' or ``bounded'' is dropped
        from the hypotheses.

    \item \textbf{Lemma 3.28} \\
        Suppose that $a < B$ and that $f : [a, b) \to \mathbb{R}$.
        If $f$ is continuous at a point $x_0 \in [a,b)$ and $f(x_0) > 0$,
        then there exist a positive number $\epsilon$ and a point
        $x_1 \in [a, b)$ such that $x_1 > x_0$ and $f(x) > \epsilon \ \forall
        x \in [x_0, x_1]$.

    \item \textbf{Theorem 3.29} \emph{Intermediate Value Theorem} \\
        Suppose that $a < b$ and that $f : [a, b] \to \mathbb{R}$ is continuous.
        If $y_0$ lies between $f(a)$ and $f(b)$, then there is an $x_0 \in (a, b)$
        such that $f(x_0) = y_0$.

    \item \textbf{Remark 3.34}
        The composition of two functions $g \circ f$ can be nowhere continuous,
        even though $f$ is discontinuous only on $\mathbb{Q}$ and $g$ is discontinuous
        at only one point.

\end{itemize}

\subsection{Uniform Continuity}

\begin{itemize}

    \item \textbf{Definition 3.35} \emph{Uniform continuity} \\
        Let $E$ be a nonempty subset of $\mathbb{R}$ and $f : E \to \mathbb{R}$.
        Then $\mathbb{f}$ is said to be \emph{uniformly continuous} on $E$ if and only if
        for every $\epsilon > 0$ there is a $\delta > 0$ such that
        \[
            |x-a| < delta \quad \text{and} \quad x, a, \in E \implies |f(x) - f(a)| < \epsilon
        \]

    \item \textbf{Lemma 3.38} \\
        Suppose that $E \subseteq \mathbb{R}$ and that $f : E \to \mathbb{R}$
        is uniformly continuous.
        If $x_n \in E$ is Cauchy, the $f(x_n)$ is Cauchy.

    \item \textbf{Theorem 3.39} \\
        Suppose that $I$ is a closed, bounded interval.
        If $f : I \to \mathbb{R}$ is continuous on $I$, then $f$ is uniformly continuous on
        $I$.

    \item \textbf{Theorem 3.40} \\
        Suppose that $a < b$ and that $f : (a, b) \to \mathbb{R}$.
        Then $f$ is uniformly continuous on $(a, b)$ if and only if $f$ can be continuously
        extended to $[a, b]$; that is, if and only if there is a continuous function
        $g : [a, b] \to \mathbb{R}$ which satisfies
        \[
            f(x) = g(x), \quad x \in (a, b)
        \]
\end{itemize}

\break{}

\section{Differentiability on \mathbb{R}}

\subsection{The Derivative}

\begin{itemize}

    \item \textbf{Definition 4.1} \emph{Differentiable} \\
        A real function $f$ is said to be \emph{differentiable} at a point $a \in \mathbb{R}$
        if and only if $f$ is defined on some open interval $I$ containing $a$ and
        \[
            f'(a) := \lim_{h \to 0} \frac{f(a+h) - f(a)}{h}
        \]
        exists.
        In this case $f'(a)$ is called the \emph{derivative} of $f$ at $a$.

    \item \textbf{Theorem 4.2} \\
        A real function $f$ is differentiable at some point $a \in \mathbb{R}$ if and only if
        there exist an open interval $I$ and a function $F : I \to \mathbb{R}$ such that
        $a \in I$, $f$ is defined on $I$, $F$ is continuous at $a$, and
        \[
            f(x) = F(x) (x-a) + f(a)
        \]
        holds for all $x \in I$ in which case $F(a) = f'(a)$.

    \item \textbf{Theorem 4.3} \\
        A real function $f$ is differentiable at $a$ if and only if there is a function
        $T$ of the form $T(x) := m(x)$ such that
        \[
            \lim_{h \to 0} \frac{f(a+h) - f(a) - t(h)}{h} = 0
        \]

    \item \textbf{Theorem 4.4} \\
        If $f$ is differentiable at $a$, then $f$ is continuous at $a$.

    \item \textbf{Definition 4.6} \emph{Continuously differentiable} \\
        Let $I$ be a nondegenerate interval.
        \begin{enumerate}
            \item A function $f : I \to \mathbb{R}$ is said to be \emph{differentiable}
                on $I$ if and only if
                \[
                    f_i'(a) := \lim_{x \to a; x \in I} \frac{f(x) - f(a)}{x-a}
                \]
                exists and is finite for every $a \in I$.
            \item $f$ is said to be \emph{continuously differentiable} on $I$ if and only if
                $f_I'$ exists and is continuous on $I$.
        \end{enumerate}

    \item \textbf{Remark 4.9} \\
        $f(x) = |x|$ is differentiable on $[0, 1]$ and on $[-1, 0]$ but not on $[-1, 1]$.

\end{itemize}

\subsection{Differentiability Theorems}

\begin{itemize}
    \item \textbf{Theorem 4.10} \\
        Let $f$ and $g$ be real functions and $\alpha \in \mathbb{R}$.
        If $f$ and $g$ are differentiable at $a$, then $f+g$, $\alpha f$, $f \cdot g$,
        and [when $g(a) \neq 0$] $f/g$ are all differentiable at a.
        In fact,
        \begin{align*}{}
            (f+g)'(a) &= f'(a) + g'(a) \\
            (\alpha f)'(a) &= \alpha f'(a) \\
            (f \cdot g)'(a) &= g(a) g'(a) + f(a) g'(a) \\
            \left( \frac{f}{g}\right)'(a) &= \frac{g(a) f'(a) - f(a) g'(a)}{g^2(a)} \\
        \end{align*}

    \item \textbf{Theorem 4.11} \emph{Chain Rule} \\
        Let $f$ and $g$ be real functions.
        If $f$ is differentiable at $a$ and $g$ is differentiable at $f(a)$, then
        $g \circ f$ is differentiable at $a$ with
        \[
            (g \circ f)'(a) = g'(f(a))f'(a)
        \]

\end{itemize}

\subsection{Mean Value Theorem}

\begin{itemize}

    \item \textbf{Lemma 4.12} \emph{Rolle's Theorem} \\
        Suppose that $a, b \in \mathbb{R}$ with $a < b$.
        If $f$ is continuous on $[a, b]$, differentiable on $(a, b)$, and if $f(a) = f(b)$,
        then $f'(c) = 0$ for some $c \in (a, b)$.

    \item \textbf{Remark 4.13} \\
        The continuity hypothesis on Rolle's Theorem cannot be relaxed at even one point
        in $[a, b]$.

    \item \textbf{Remark 4.14} \\
        The differentiability hypothesis on Rolle's Theorem cannot be relaxed at even one point
        in $[a, b]$.

    \item \textbf{Theorem 4.15} \\
        Suppose that $a, b, \in \mathbb{R}$ with $a<b$.
        \begin{enumerate}
            \item \emph{Generalised Mean Value Theorem}:
                If $f$, $g$ are continuous on $[a, b]$ and differentiable on $(a, b)$,
                then there is a $c \in (a, b)$ such that
                \[
                    g'(c) (f(b) - f(a)) = f'(c) (g(b) - g(a))
                \]

            \item \emph{Mean Value Theorem}:
                If $f$ is continuous on $[a, b]$ and differentiable on $(a, b)$, then there
                is a $c \in (a, b)$ such that
                \[
                    f(b) - f(a) = f'(c)(b-A)
                \]

        \end{enumerate}

    \item \textbf{Definition 4.16} \emph{Increasing, Monotone, Decreasing} \\
        Let $E$ be a nonempty subset of $\mathbb{R}$ and $f : E \to \mathbb{R}$.
        \begin{enumerate}
            \item $f$ is said to be \emph{increasing}
                (respectively, \emph{strictly increasing}) on $E$ if and only if
                $x_1, x_2 \in E$ and $x_1 < x_2 \implies f(x_1) \leq f(x_2)$
                [respectively, $f(x_1) < f(x_2)$].
            \item $f$ is said to be \emph{decreasing}
                (respectively, \emph{strictly decreasing}) on $E$ if and only if
                $x_1, x_2 \in E$ and $x_1 < x_2 \implies f(x_1) \geq f(x_2)$
                [respectively, $f(x_1) > f(x_2)$].
            \item $f$ is said to be \emph{monotone} (respectively, \emph{strictly monotone})
                on $E$ if and only if $f$ is either decreasing or increasing
                (respectively, either strictly decreasing or strictly increasing) on $E$.
        \end{enumerate}

    \item \textbf{Theorem 4.17} \\
        Suppose that $a, b \in \mathbb{R}$, with $a<b$, that $f$ is continuous on $[a, b]$,
        and that $f$ is differentiable on $(a, b)$.
        \begin{enumerate}
            \item If $f'(x) > 0$ [respectively $f'(x) < 0$] for all $x \in (a, b)$,
                then $f$ is strictly increasing (respectively, strictly decreasing) on
                $[a, b]$.
            \item If $f'(x) = 0$ for all $x \in (a, b)$, then $f$ is constant on $[a, b]$.
            \item If $g$ is continuous on $[a, b]$ and differentiable on $(a, b)$,
                and if $f'(x) = g'(x)$ for all $x \in (a, b)$, then $f-g$ is constant on
                $[a, b]$.
        \end{enumerate}

    \item \textbf{Theorem 4.18} \\
        Suppose that $f$ is increasing on $[a, b]$
        \begin{enumerate}
            \item If $c \in [a, b)$, then $f(c+)$ exists and $f(c) \leq f(c+)$.
            \item If $c \in (a, b]$, then $f(c-)$ exists and $f(c-) \leq f(c)$.
        \end{enumerate}

    \item \textbf{Theorem 4.19} \\
        If $f$ is monotone on an interval $I$, then $f$ has at most countable many points
        of discontinuity on $I$.

    \item \textbf{Theorem 4.21} \emph{Bernoulli's Inequality} \\
        Let $\alpha$ be a positive real number.
        If $0 < \alpha < 1$, then
        ${(1+x)}^\alpha \leq 1 + \alpha x \ \forall x \in [-1, \infty)$,
        and if $\alpha \geq 1$, then
        ${(1+x)}^\alpha \geq 1 + \alpha x \ \forall x \in [-1, \infty)$.

    \item \textbf{Theorem 4.23} \emph{Intermediate Value Theorem for Derivatives} \\
        Suppose that $f$ is differentiable on $[a, b]$ with $f'(a) \neq f'(b)$.
        If $y_0$ is a real number which lies between $f'(a)$ and $f'(b)$,
        then there is an $x_0 \in (a, b)$ such that $f'(x_0) = y_0$.
\end{itemize}

\subsection{Taylor's Theorem and L'Hopital's Rule}

\begin{itemize}

    \item \textbf{Theorem 4.24} \emph{Taylor's Formula} \\
        Let $n \in \mathbb{N}$ and let $a, b$ be extended real numbers with $a<b$.
        If $f : (a, b) \to \mathbb{R}$, and if $f^{(n+1)}$ exists on $(a, b)$,
        then for each pair of points $(x, x_0 \in (a, b)$ there is a number $c$
        between $x$ and $x_0$ such that
        \[
            f(x) = f(x_0) + \sum_{k=1}^n \frac{f^{(k)} (x_0)}{k!}(x-x_0)^k +
            \frac{f^{(n+1)}(c)}{(n+1)!} (x-x_0)^{n+1}
        \]

    \item \textbf{Theorem 4.27} \emph{L'Hopital's Rule} \\
        Let $a$ be an extended real number and $I$ be an open interval which either contains
        $a$ or has $a$ as an endpoint.
        Suppose that $f$ and $g$ are differentiable on $I \setminus \{a\}$ and that
        $g(x) \neq 0 \neq g'(x) \ \forall x \in I \setminus \{a\}$.
        Suppose further that
        \[
            A := \lim_{x \to a; x \in I} f(x) = \lim_{x \to a; x \in I} g(x)
        \]
        is either $0$ or $\infty$.
        If
        \[
            B := \lim_{x \to a; x \in I} \frac{f'(x)}{g'(x)}
        \]
        exists as an extended real number, then
        \[
            \lim_{x \to a; x \in I} \frac{f(x)}{g(x)} =
            \lim_{x \to a; x \in I} \frac{f'(x)}{g'(x)}
        \]

\end{itemize}

\subsection{Inverse Function Theorems}

\begin{itemize}

    \item \textbf{Theorem 4.32} \\
        Let $I$ be a nondegenerate interval and suppose that $f : I \to \mathbb{R}$ is
        injective.
        If $f$ is continuous on $I$, then $J := f(I)$ is an interval, $f$ is strictly
        monotone on $I$, and $f^{-1}$ is continuous and strictly monotone on $J$.

    \item \textbf{Theorem 4.33} \emph{Inverse Function Theorem} \\
        Let $I$ be an open interval and $f : I \to \mathbb{R}$ be injective and continuous.
        If $b = f(a)$ for some $a \in I$ and if $f'(a)$ exists and is nonzero,
        then $f^{-1}$ is differentiable at $b$ and $(f^{-1})'(b) = \frac{1}{f'(a)}$.
\end{itemize}<++>

\end{document}
