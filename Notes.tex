\documentclass[11pt,a4paper]{article}

\usepackage[utf8]{inputenc}
\usepackage{parskip}
\usepackage{tabularx}
\usepackage{amsmath}
\usepackage{amssymb}
\usepackage{amsthm}
\usepackage{geometry}
\usepackage{booktabs}
\usepackage{centernot}
\usepackage{hyperref}
\usepackage{eufrak}
\usepackage{graphicx}
\graphicspath{{pics/}}
\geometry{a4paper, left=20mm, right=20mm, top=20mm, bottom=20mm}

\usepackage{fancyhdr}
\pagestyle{fancy}
\lhead{Anthony Catterwell}
\chead{\textsc{University of Edinburgh}}
\rhead{Analysis}

\title{Honours Analysis Notes}
\author{Anthony Catterwell}

\pagenumbering{gobble}

\begin{document}
\maketitle
\tableofcontents

\break{}

\section{The Real Number System}

\subsection{Introduction}

\subsection{Ordered Field Axioms}

\begin{itemize}
    \item \textbf{Postulate 1} \emph{Field Axioms} \\
        There are functions $+$ and $\cdot$ defined on
        $\mathbb{R}^2 := \mathbb{R} \times \mathbb{R}$,
        which satisfy the following properties $\forall a, b, c \in \mathbb{R}$

        \begin{itemize}
            \item \emph{Closure Properties}:
                $a + b, \ a \cdot b \in \mathbb{R}$

            \item \emph{Associative Properties}:
                $a + (b + c) = (a + b) + c$ and
                $a \cdot (b \cdot c) = (a \cdot b) \cdot c$

            \item \emph{Commutative Properties}:
                $a + b = b + a$ and $a \cdot b = b \cdot a$

            \item \emph{Distributive Law}:
                $a \cdot (b + c) = a \cdot b + a \cdot c$

            \item \emph{Existence of Additive Identity}:
                There is a unique element $0 \in \mathbb{R}$ such that
                $0 + a = a$ for all $a \in \mathbb{R}$

            \item \emph{Existence of Multiplicative Identity}:
                There is a unique element $1 \in \mathbb{R}$ such that
                $1 \neq 0$ and $1 \cdot a = a$ for all $a \in \mathbb{R}$

            \item \emph{Existence of Additive Inverses}:
                For every $x \in \mathbb{R}$ there is a unique element $-x \in \mathbb{R}$ such that
                \[
                    x + (-x) = 0
                \]

            \item \emph{Existence of Multiplicative Inverses}:
                For every $x \in \mathbb{R} \setminus \{0\}$ there is a unique element
                $x^{-1} \in \mathbb{R}$ such that
                \[
                    x \cdot (x^{-1}) = 1
                \]
        \end{itemize}

    \item \textbf{Postulate 2} \emph{Order Axioms} \\
        There is a relation $<$ on $\mathbb{R} \times \mathbb{R}$ that has the following
        properties:
        \begin{itemize}
            \item \emph{Trichotomy Property}:
                Given $a, b \in \mathbb{R}$, one and only one of the following statements hold:
                \[
                    a < b, \ b < a, \ \text{or} \ a = b
                \]

            \item \emph{Transitive Property}:
                For $a, b, c, \in \mathbb{R}$
                \[
                    a < b \ \text{and} \b < c \implies a < c
                \]

            \item \emph{Additive Property}:
                For $a, b, c \in \mathbb{R}$
                \[
                    a < b \ \text{and} \ c \in \mathbb{R} \implies a + c < b + c
                \]

            \item \emph{Multiplicative Properties}:
                For $a, b, c, \in \mathbb{R}$
                \[
                    a < b \ \text{and} \ c > 0 \implies ac < bc
                \]
                and
                \[
                    a < b \ \text{and} \ c < 0 \implies bc < ac
                \]
        \end{itemize}

    \item \textbf{Remark 1.1} \\
        We will assume that the sets $\mathbb{N}$ and $\mathbb{Z}$ satisfy the following properties:
        \begin{enumerate}
            \item If $n, m \in \mathbb{Z}$, then $n + m, n - m$ and $mn$ belong to $\mathbb{Z}$
            \item If $n \in \mathbb{Z}$, then $n \in \mathbb{N}$ if and only if $n \geq 1$
            \item There is no $n \in \mathbb{Z}$ that satisfies $0 < n < 1$
        \end{enumerate}

    \item \textbf{Definition 1.4} \emph{Absolute Value} \\
        The \emph{absolute value} of a number $a \in \mathbb{R}$ is the number
        \[
            |a| :=
            \begin{cases}{}
                a  & a \geq 0 \\
                -a & a < 0 \\
            \end{cases}
        \]

    \item \textbf{Remark 1.5} The \emph{absolute value} is multiplicative;
        that is, $|ab| = |a||b| \ \forall a, b, \in \mathbb{R}$

    \item \textbf{Theorem 1.6} \emph{Fundamental Theorem of Absolute Values} \\
        Let $a \in \mathbb{R}$ and $M \geq 0$.
        Then $|a| \leq M \iff -M \leq a \leq M$.

    \item \textbf{Theorem 1.7} The absolute value satisfies the following three properties:
        \begin{enumerate}
            \item \emph{Positive Definite}:
                For all $a \in \mathbb{R}$, $|a| > 0$ with $|a| = 0$ if and only if $a = 0$.
            \item \emph{Symmetric}:
                For all $a, b, \in \mathbb{R}$, $|a - b| = |b - a|$,
            \item \emph{Triangle Inequalities}:
                For all $a, b \in \mathbb{R}$,
                \[
                    |a + b| \leq |a| + |b| \quad \text{and}
                    \quad \lvert |a| - |b| \rvert \leq |a - b|
                \]
        \end{enumerate}

    \item \textbf{Theorem 1.9} Let $x, y, a \in \mathbb{R}$
        \begin{enumerate}
            \item $x < y + \epsilon \ \forall \epsilon > 0 \iff x \leq y$
            \item $x > y - \epsilon \ \forall \epsilon > 0 \iff x \geq y$
            \item $|a| < \epsilon \ \forall \epsilon > 0 \iff a = 0$
        \end{enumerate}

\end{itemize}

\subsection{Completeness Axiom}

\begin{itemize}
    \item \textbf{Definition 1.10} \emph{Upper bounds} \\
        Let $E \subset \mathbb{R}$ be non-empty
        \begin{enumerate}
            \item The set $E$ is said to be \emph{bounded above} if and only if there is an
                $M \in \mathbb{R}$ such that $a \leq M$ for all $a \in E$,
                in which case $M$ is called an \emph{upper bound} of $E$.
            \item A number $s$ is called a \emph{supremum} of the set $E$ if and only if $s$
                is an upper bound of $E$ and $s \leq M$ for all upper bounds $M$ of $E$.
                (In this case we shall say that $E$ has a \emph{finite supremeum} $s$ and write
                $s = \sup E$)
        \end{enumerate}

    \item \textbf{Remark 1.12}
        If a set has one upper bound, it has infinitely many upper bounds.

    \item \textbf{Remark 1.13}
        If a set has a supremum, then it has only one supremum.

    \item \textbf{Theorem} \emph{Approximation Property for Suprema} \\
        If $E$ has a finite supremum and $\epsilon > 0$ is any positive number, then there is a
        point $a \in E$ such that
        \[
            \sup E - \epsilon < a \leq \sup E
        \]

    \item \textbf{Theorem 1.15} \\
        If $E \subset \mathbb{Z}$ has a supremum, then $\sup E \in E$.
        In particular, if the supremum of a set, which contains only integers, exists,
        that supremum must be an integer.

    \item \textbf{Postulate 3} \emph{Completeness Axiom} \\
        If $E$ is a nonempty subset of $\mathbb{R}$ that is bounded above, then $E$ has a finite
        supremum.

    \item \textbf{Theorem 1.16} \emph{The Archimedean Principle} \\
        Given real numbers $a$ and $b$, with $a > 0$, there is an integer $n \in \mathbb{N}$
        such that $b < na$.

    \item \textbf{Theorem 1.18} \emph{Density of Rationals} \\
        If $a, b, \in \mathbb{R}$ satisfy $a < b$, then there is a $q \in \mathbb{Q}$
        such that $a < q < b$.

    \item \textbf{Definition 1.19} \emph{Upper bounds} \\
        Let $E \in \mathbb{R}$ be nonempty
        \begin{enumerate}
            \item The set $E$ is said to be \emph{bounded below} if and only if there is an
                $m \in \mathbb{R}$ such that $a \geq E$, in which case $m$ is called a
                \emph{lower bound} of the set $E$.
            \item A number $t$ is called an \emph{infimum} of the set $E$ if and only if $t$
                is a lower bound of $E$ and $t \geq m$ and write $t = \inf E$.
            \item $E$ is said to be \emph{bounded} if and only if it is bounded both above and
                below.
        \end{enumerate}

    \item \textbf{Theorem 1.20} \emph{Reflection Principle} \\
        Let $E \in \mathbb{R}$ be nonempty
        \begin{enumerate}
            \item $E$ has a supremum if and only if $-E$ has an infimum, in which case
                \[
                    \inf (-E) = -\sup E
                \]
            \item $E$ has an infimum if and only if $-E$ has a supremum, in which case
                \[
                    \sup (-E) = -\inf E
                \]
        \end{enumerate}

    \item \textbf{Theorem 1.21} \emph{Monotone Property} \\
        Suppose that $A \subseteq B$ are nonempty subsets of $\mathbb{R}$.
        \begin{enumerate}
            \item If $B$ has a supremum, then $\sup A \leq \sup B$.
            \item If $B$ has an infimum, then $\inf A \geq \inf B$.
        \end{enumerate}

\end{itemize}

\subsection{Mathematical Induction}

\begin{itemize}
    \item \textbf{Theorem 1.22} \emph{Well-Ordering Principle} \\
        If $E$ is a nonempty subset of $\mathbb{N}$, then $E$ has a least element
        (i.e.\ $E$ has a finite infimum and $\inf E \in E$).

    \item \textbf{Theorem 1.23} \\
        Suppose for each $n \in \mathbb{N}$ that $A(n)$ is a proposition which satisfies
        the following two properties:
        \begin{enumerate}
            \item $A(1)$ is true.
            \item For every $n \in \mathbb{N}$ for which $A(n)$ is true, $A(n+1)$ is also true.
        \end{enumerate}
        Then $A(n)$ is true for all $n \in \mathbb{N}$.

    \item \textbf{Theorem 1.26} \emph{Binomial Formula} \\
        If $a, b, \in \mathbb{R}, n \in \mathbb{N}$ and $0^0$ is interpreted to be $1$, then
        \[
            {(a+b)}^n = \sum_{k=0}^n
            \begin{pmatrix}
                n \\ k
            \end{pmatrix}
            a^{n-k} b^k
        \]

\end{itemize}

\subsection{Inverse Functions and Images}

\begin{itemize}
    \item \textbf{Definition 1.29} \emph{Injection, Surjection, Bijection} \\
        Let $X$ and $Y$ be sets and $f : X \to Y$
        \begin{enumerate}
            \item $f$ is said to be \emph{injective} if and only if
                \[
                    x_1, x_2 \in X \ \text{and} \ f(x_1) = f(x_2) \implies x_1 = x_2
                \]
            \item $f$ is said to be \emph{surjective} if and only if
                \[
                    \forall y \in Y \ \exists x \in X \backepsilon y = f(x)
                \]

            \item $f$ is called \emph{bijective} if and only if it is both injective
                and surjective
        \end{enumerate}

    \item \textbf{Theorem 1.30} \\
        Let $X$ and $Y$ be sets and $f : X \to Y$.
        Then the following three statements are equivalent.
        \begin{enumerate}
            \item $f$ has an inverse;
            \item $f$ is injective from $X$ onto $Y$;
            \item There is a function $g : Y \to X$ such that
                \[
                    g(f(x)) = x \quad \forall x \in X
                \]
                and
                \[
                    f(g(y)) = y \quad \forall y \in Y
                \]
        \end{enumerate}
        Moreover, for each $f : X \to Y$, there is only one function $g$ that satisfies these.
        It is the inverse function $f^{-1}$.

    \item \textbf{Remark 1.31} \\
        Let $I$ be an interval and let $f : I \to \mathbb{R}$.
        If the derivative of $f$ is either always positive on $I$, or always negative on $I$,
        then $f$ is injective on $I$.

    \item \textbf{Definition 1.33} \emph{Image} \\
        Let $X$ and $Y$ be sets and $f : X \to Y$.
        The \emph{image} of a set $E \subseteq X$ under $f$ is the set
        \[
            f(E) := \{ y \in Y : y = f(x) \ \text{for some} \ x \in E \}
        \]
        The \emph{inverse image} of a set $E \subseteq Y$ under $f$ is the set
        \[
            f^{-1}(E) := \{ x \in X : f(x) = y \ \text{for some} \ y \in E \}
        \]

    \item \textbf{Definition 1.35} \emph{Union, Intersection} \\
        Let $\mathcal{E} = {\{ E_{\alpha} \}}_{\alpha \in A}$ be a collection of sets.
        \begin{enumerate}
            \item The \emph{union} of the collection $\mathcal{E}$ is the set
                \[
                    \bigcup_{\alpha \in A} E_\alpha := \{ x : x \in E_\alpha \ \text{for some} \
                    \alpha \in A \}
                \]
            \item The \emph{intersection} of the collection $\mathcal{E}$ is the set
                \[
                    \bigcap_{\alpha \in A} E_\alpha := \{ x : x \in E_\alpha \ \text{for all} \
                    \alpha \in A \}.
                \]
        \end{enumerate}

    \item \textbf{Theorem 1.36} \emph{DeMorgan's Laws} \\
        Let $X$ be a set and ${\{ E_\alpha \}}_{\alpha \in A}$ be a collection of subsets of $X$.
        If for each $E \subseteq X$ the symbol $E^c$ represents the set $X \setminus E$, then
        \[
            {\left( \bigcup_{\alpha \in A} E_\alpha \right)}^c =
            \bigcap_{\alpha \in A} E_\alpha^c
        \]
        and
        \[
            {\left( \bigcap_{\alpha \in A} E_\alpha \right)}^c =
            \bigcup_{\alpha \in A} E_\alpha ^ c
        \]

    \item \textbf{Theorem 1.37} \\
        Let $X$ and $Y$ be sets and $f : X \to Y$.
        \begin{enumerate}
            \item If ${\{ E_\alpha \}}{\alpha \in A}$ is a collection of subsets of $X$, then
                \[
                    f \left( \bigcup_{\alpha \in A} E_\alpha \right) =
                    \bigcup_{\alpha \in A} f(E_\alpha)
                    \quad \text{and} \quad
                    f \left( \bigcap_{\alpha \in A} E_\alpha \right) \subseteq
                    \bigcap_{\alpha \in A} f(E_\alpha)
                \]
            \item If $B$ and $C$ are subsets of $X$, then
                $f(C) \setminus f(B) \subseteq f(C \setminus B)$
            \item If ${\{E_\alpha\}}_{\alpha \in A}$ is a collection of subsets of $Y$, then
                \[
                    f^{-1} \left( \bigcup_{\alpha \in A} E_\alpha \right) =
                    \bigcup_{\alpha \in A} f^{-1} (E_\alpha)
                    \quad \text{and} \quad
                    f^{-1} \left( \bigcap_{\alpha \in A} E_\alpha \right) =
                    \bigcap_{\alpha \in A} f^{-1} (E_\alpha)
                \]
            \item If $B$ and $C$ are subsets of $Y$, then
                $f^{-1}(C \setminus B) = f^{-1}(C) \setminus f^{-1}(B)$.
            \item If $E \subseteq f(x)$, then $f(f^{-1}(E)) = E$, but if $E \subseteq X$,
                then $E \subseteq f^{-1}(f(E))$.
        \end{enumerate}

\end{itemize}

\subsection{Countable and Uncountable Sets}

\begin{itemize}
    \item \textbf{Definition 1.38} \emph{Countable & Uncountable}
        Let $E$ be a set.
        \begin{enumerate}
            \item $E$ is said to be \emph{finite} if and only if either $E = \emptyset$
                or there exists an injective function which takes
                $\{ 1, 2, \ldots, n \}$ onto $E$, for some $n \in \mathbb{N}$.
            \item $E$ is said to be \emph{countable} if and only if there exists and injective
                function which takes $\mathbb{N}$ onto $E$.
            \item $E$ is said to be \emph{at most countable} if and only if $E$ is either
                finite or countable.
            \item $E$ is said to be \emph{uncountable} if and only if $E$ is neither finite nor
                countable.
        \end{enumerate}

    \item \textbf{Remark 1.39} \emph{Cantor's Diagonalisation Argument} \\
        The open interval $(0, 1)$ is uncountable.

    \item \textbf{Lemma 1.40} \\
        A nonempty set $E$ is at most countable if and only if there is a function $g$ from
        $\mathbb{N}$ onto $E$.

    \item \textbf{Theorem 1.41} \\
        Suppose $A$ and $B$ are sets.
        \begin{enumerate}
            \item If $A \subseteq B$ and $B$ is at most countable, then $A$ is at most countable.
            \item If $A \subseteq B$ and $A$ is uncountable, then $B$ is uncountable.
            \item $\mathbb{R}$ is uncountable.
        \end{enumerate}

    \item \textbf{Theorem 1.42} \\
        Let $A_1, A_2, \ldots$ be at most countable sets.
        \begin{enumerate}
            \item Then $A_1 \times A_2$ is at most countable.
            \item If
                \[
                    E = \bigcup_{j=1}^\infty A_j :=
                    \bigcup_{j \in \mathbb{N}} A_j :=
                    \{ x : x \in A_j \ \text{for some} \ j \in \mathbb{N} \},
                \]
                then $E$ is at most countable.
        \end{enumerate}

    \item \textbf{Remark 1.43} \\
        The sets $\mathbb{Z}$ and $\mathbb{Q}$ are countable, but the set of irrationals is
        uncountable.

\end{itemize}

\end{document}
