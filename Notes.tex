\documentclass[11pt,a4paper]{article}

\usepackage[utf8]{inputenc}
\usepackage{parskip}
\usepackage{tabularx}
\usepackage{amsmath}
\usepackage{amssymb}
\usepackage{amsthm}
\usepackage{geometry}
\usepackage{booktabs}
\usepackage{centernot}
\usepackage{hyperref}
\usepackage{eufrak}
\usepackage{graphicx}
\graphicspath{{pics/}}
\geometry{a4paper, left=20mm, right=20mm, top=20mm, bottom=20mm}

\usepackage{fancyhdr}
\pagestyle{fancy}
\lhead{Anthony Catterwell}
\chead{\textsc{University of Edinburgh}}
\rhead{Analysis}

\title{Honours Analysis Notes}
\author{Anthony Catterwell}

\pagenumbering{gobble}

\begin{document}
\maketitle
\tableofcontents

\break{}

\section{The Real Number System}

\subsection{Introduction}

\subsection{Ordered Field Axioms}

\begin{itemize}
    \item \textbf{Postulate 1} \emph{Field Axioms} \\
        There are functions $+$ and $\cdot$ defined on
        $\mathbb{R}^2 := \mathbb{R} \times \mathbb{R}$,
        which satisfy the following properties $\forall a, b, c \in \mathbb{R}$

        \begin{itemize}
            \item \emph{Closure Properties}:
                $a + b, \ a \cdot b \in \mathbb{R}$

            \item \emph{Associative Properties}:
                $a + (b + c) = (a + b) + c$ and
                $a \cdot (b \cdot c) = (a \cdot b) \cdot c$

            \item \emph{Commutative Properties}:
                $a + b = b + a$ and $a \cdot b = b \cdot a$

            \item \emph{Distributive Law}:
                $a \cdot (b + c) = a \cdot b + a \cdot c$

            \item \emph{Existence of Additive Identity}:
                There is a unique element $0 \in \mathbb{R}$ such that
                $0 + a = a$ for all $a \in \mathbb{R}$

            \item \emph{Existence of Multiplicative Identity}:
                There is a unique element $1 \in \mathbb{R}$ such that
                $1 \neq 0$ and $1 \cdot a = a$ for all $a \in \mathbb{R}$

            \item \emph{Existence of Additive Inverses}:
                For every $x \in \mathbb{R}$ there is a unique element $-x \in \mathbb{R}$ such that
                \[
                    x + (-x) = 0
                \]

            \item \emph{Existence of Multiplicative Inverses}:
                For every $x \in \mathbb{R} \setminus \{0\}$ there is a unique element
                $x^{-1} \in \mathbb{R}$ such that
                \[
                    x \cdot (x^{-1}) = 1
                \]
        \end{itemize}

    \item \textbf{Postulate 2} \emph{Order Axioms} \\
        There is a relation $<$ on $\mathbb{R} \times \mathbb{R}$ that has the following
        properties:
        \begin{itemize}
            \item \emph{Trichotomy Property}:
                Given $a, b \in \mathbb{R}$, one and only one of the following statements hold:
                \[
                    a < b, \ b < a, \ \text{or} \ a = b
                \]

            \item \emph{Transitive Property}:
                For $a, b, c, \in \mathbb{R}$
                \[
                    a < b \ \text{and} \b < c \implies a < c
                \]

            \item \emph{Additive Property}:
                For $a, b, c \in \mathbb{R}$
                \[
                    a < b \ \text{and} \ c \in \mathbb{R} \implies a + c < b + c
                \]

            \item \emph{Multiplicative Properties}:
                For $a, b, c, \in \mathbb{R}$
                \[
                    a < b \ \text{and} \ c > 0 \implies ac < bc
                \]
                and
                \[
                    a < b \ \text{and} \ c < 0 \implies bc < ac
                \]
        \end{itemize}

    \item \textbf{Remark 1.1} \\
        We will assume that the sets $\mathbb{N}$ and $\mathbb{Z}$ satisfy the following properties:
        \begin{enumerate}
            \item If $n, m \in \mathbb{Z}$, then $n + m, n - m$ and $mn$ belong to $\mathbb{Z}$
            \item If $n \in \mathbb{Z}$, then $n \in \mathbb{N}$ if and only if $n \geq 1$
            \item There is no $n \in \mathbb{Z}$ that satisfies $0 < n < 1$
        \end{enumerate}

    \item \textbf{Definition 1.4} \emph{Absolute Value} \\
        The \emph{absolute value} of a number $a \in \mathbb{R}$ is the number
        \[
            |a| :=
            \begin{cases}{}
                a  & a \geq 0 \\
                -a & a < 0 \\
            \end{cases}
        \]

    \item \textbf{Remark 1.5} The \emph{absolute value} is multiplicative;
        that is, $|ab| = |a||b| \ \forall a, b, \in \mathbb{R}$

    \item \textbf{Theorem 1.6} \emph{Fundamental Theorem of Absolute Values} \\
        Let $a \in \mathbb{R}$ and $M \geq 0$.
        Then $|a| \leq M \iff -M \leq a \leq M$.

    \item \textbf{Theorem 1.7} The absolute value satisfies the following three properties:
        \begin{enumerate}
            \item \emph{Positive Definite}:
                For all $a \in \mathbb{R}$, $|a| > 0$ with $|a| = 0$ if and only if $a = 0$.
            \item \emph{Symmetric}:
                For all $a, b, \in \mathbb{R}$, $|a - b| = |b - a|$,
            \item \emph{Triangle Inequalities}:
                For all $a, b \in \mathbb{R}$,
                \[
                    |a + b| \leq |a| + |b| \quad \text{and}
                    \quad \lvert |a| - |b| \rvert \leq |a - b|
                \]
        \end{enumerate}

    \item \textbf{Theorem 1.9} Let $x, y, a \in \mathbb{R}$
        \begin{enumerate}
            \item $x < y + \epsilon \ \forall \epsilon > 0 \iff x \leq y$
            \item $x > y - \epsilon \ \forall \epsilon > 0 \iff x \geq y$
            \item $|a| < \epsilon \ \forall \epsilon > 0 \iff a = 0$
        \end{enumerate}

\end{itemize}

\subsection{Completeness Axiom}

\begin{itemize}
    \item \textbf{Definition 1.10} \emph{Upper bounds} \\
        Let $E \subset \mathbb{R}$ be non-empty
        \begin{enumerate}
            \item The set $E$ is said to be \emph{bounded above} if and only if there is an
                $M \in \mathbb{R}$ such that $a \leq M$ for all $a \in E$,
                in which case $M$ is called an \emph{upper bound} of $E$.
            \item A number $s$ is called a \emph{supremum} of the set $E$ if and only if $s$
                is an upper bound of $E$ and $s \leq M$ for all upper bounds $M$ of $E$.
                (In this case we shall say that $E$ has a \emph{finite supremeum} $s$ and write
                $s = \sup E$)
        \end{enumerate}

    \item \textbf{Remark 1.12}
        If a set has one upper bound, it has infinitely many upper bounds.

    \item \textbf{Remark 1.13}
        If a set has a supremum, then it has only one supremum.

    \item \textbf{Theorem} \emph{Approximation Property for Suprema} \\
        If $E$ has a finite supremum and $\epsilon > 0$ is any positive number, then there is a
        point $a \in E$ such that
        \[
            \sup E - \epsilon < a \leq \sup E
        \]

    \item \textbf{Theorem 1.15} \\
        If $E \subset \mathbb{Z}$ has a supremum, then $\sup E \in E$.
        In particular, if the supremum of a set, which contains only integers, exists,
        that supremum must be an integer.

    \item \textbf{Postulate 3} \emph{Completeness Axiom} \\
        If $E$ is a nonempty subset of $\mathbb{R}$ that is bounded above, then $E$ has a finite
        supremum.

    \item \textbf{Theorem 1.16} \emph{The Archimedean Principle} \\
        Given real numbers $a$ and $b$, with $a > 0$, there is an integer $n \in \mathbb{N}$
        such that $b < na$.

    \item \textbf{Theorem 1.18} \emph{Density of Rationals} \\
        If $a, b, \in \mathbb{R}$ satisfy $a < b$, then there is a $q \in \mathbb{Q}$
        such that $a < q < b$.

    \item \textbf{Definition 1.19} \emph{Upper bounds} \\
        Let $E \in \mathbb{R}$ be nonempty
        \begin{enumerate}
            \item The set $E$ is said to be \emph{bounded below} if and only if there is an
                $m \in \mathbb{R}$ such that $a \geq E$, in which case $m$ is called a
                \emph{lower bound} of the set $E$.
            \item A number $t$ is called an \emph{infimum} of the set $E$ if and only if $t$
                is a lower bound of $E$ and $t \geq m$ and write $t = \inf E$.
            \item $E$ is said to be \emph{bounded} if and only if it is bounded both above and
                below.
        \end{enumerate}

    \item \textbf{Theorem 1.20} \emph{Reflection Principle} \\
        Let $E \in \mathbb{R}$ be nonempty
        \begin{enumerate}
            \item $E$ has a supremum if and only if $-E$ has an infimum, in which case
                \[
                    \inf (-E) = -\sup E
                \]
            \item $E$ has an infimum if and only if $-E$ has a supremum, in which case
                \[
                    \sup (-E) = -\inf E
                \]
        \end{enumerate}

    \item \textbf{Theorem 1.21} \emph{Monotone Property} \\
        Suppose that $A \subseteq B$ are nonempty subsets of $\mathbb{R}$.
        \begin{enumerate}
            \item If $B$ has a supremum, then $\sup A \leq \sup B$.
            \item If $B$ has an infimum, then $\inf A \geq \inf B$.
        \end{enumerate}

\end{itemize}

\subsection{Mathematical Induction}

\begin{itemize}
    \item \textbf{Theorem 1.22} \emph{Well-Ordering Principle} \\
        If $E$ is a nonempty subset of $\mathbb{N}$, then $E$ has a least element
        (i.e.\ $E$ has a finite infimum and $\inf E \in E$).

    \item \textbf{Theorem 1.23} \\
        Suppose for each $n \in \mathbb{N}$ that $A(n)$ is a proposition which satisfies
        the following two properties:
        \begin{enumerate}
            \item $A(1)$ is true.
            \item For every $n \in \mathbb{N}$ for which $A(n)$ is true, $A(n+1)$ is also true.
        \end{enumerate}
        Then $A(n)$ is true for all $n \in \mathbb{N}$.

    \item \textbf{Theorem 1.26} \emph{Binomial Formula} \\
        If $a, b, \in \mathbb{R}, n \in \mathbb{N}$ and $0^0$ is interpreted to be $1$, then
        \[
            {(a+b)}^n = \sum_{k=0}^n
            \begin{pmatrix}
                n \\ k
            \end{pmatrix}
            a^{n-k} b^k
        \]

\end{itemize}

\subsection{Inverse Functions and Images}

\begin{itemize}
    \item \textbf{Definition 1.29} \emph{Injection, Surjection, Bijection} \\
        Let $X$ and $Y$ be sets and $f : X \to Y$
        \begin{enumerate}
            \item $f$ is said to be \emph{injective} if and only if
                \[
                    x_1, x_2 \in X \ \text{and} \ f(x_1) = f(x_2) \implies x_1 = x_2
                \]
            \item $f$ is said to be \emph{surjective} if and only if
                \[
                    \forall y \in Y \ \exists x \in X \backepsilon y = f(x)
                \]

            \item $f$ is called \emph{bijective} if and only if it is both injective
                and surjective
        \end{enumerate}

    \item \textbf{Theorem 1.30} \\
        Let $X$ and $Y$ be sets and $f : X \to Y$.
        Then the following three statements are equivalent.
        \begin{enumerate}
            \item $f$ has an inverse;
            \item $f$ is injective from $X$ onto $Y$;
            \item There is a function $g : Y \to X$ such that
                \[
                    g(f(x)) = x \quad \forall x \in X
                \]
                and
                \[
                    f(g(y)) = y \quad \forall y \in Y
                \]
        \end{enumerate}
        Moreover, for each $f : X \to Y$, there is only one function $g$ that satisfies these.
        It is the inverse function $f^{-1}$.

    \item \textbf{Remark 1.31} \\
        Let $I$ be an interval and let $f : I \to \mathbb{R}$.
        If the derivative of $f$ is either always positive on $I$, or always negative on $I$,
        then $f$ is injective on $I$.

    \item \textbf{Definition 1.33} \emph{Image} \\
        Let $X$ and $Y$ be sets and $f : X \to Y$.
        The \emph{image} of a set $E \subseteq X$ under $f$ is the set
        \[
            f(E) := \{ y \in Y : y = f(x) \ \text{for some} \ x \in E \}
        \]
        The \emph{inverse image} of a set $E \subseteq Y$ under $f$ is the set
        \[
            f^{-1}(E) := \{ x \in X : f(x) = y \ \text{for some} \ y \in E \}
        \]

    \item \textbf{Definition 1.35} \emph{Union, Intersection} \\
        Let $\mathcal{E} = {\{ E_{\alpha} \}}_{\alpha \in A}$ be a collection of sets.
        \begin{enumerate}
            \item The \emph{union} of the collection $\mathcal{E}$ is the set
                \[
                    \bigcup_{\alpha \in A} E_\alpha := \{ x : x \in E_\alpha \ \text{for some} \
                    \alpha \in A \}
                \]
            \item The \emph{intersection} of the collection $\mathcal{E}$ is the set
                \[
                    \bigcap_{\alpha \in A} E_\alpha := \{ x : x \in E_\alpha \ \text{for all} \
                    \alpha \in A \}.
                \]
        \end{enumerate}

    \item \textbf{Theorem 1.36} \emph{DeMorgan's Laws} \\
        Let $X$ be a set and ${\{ E_\alpha \}}_{\alpha \in A}$ be a collection of subsets of $X$.
        If for each $E \subseteq X$ the symbol $E^c$ represents the set $X \setminus E$, then
        \[
            {\left( \bigcup_{\alpha \in A} E_\alpha \right)}^c =
            \bigcap_{\alpha \in A} E_\alpha^c
        \]
        and
        \[
            {\left( \bigcap_{\alpha \in A} E_\alpha \right)}^c =
            \bigcup_{\alpha \in A} E_\alpha ^ c
        \]

    \item \textbf{Theorem 1.37} \\
        Let $X$ and $Y$ be sets and $f : X \to Y$.
        \begin{enumerate}
            \item If ${\{ E_\alpha \}}{\alpha \in A}$ is a collection of subsets of $X$, then
                \[
                    f \left( \bigcup_{\alpha \in A} E_\alpha \right) =
                    \bigcup_{\alpha \in A} f(E_\alpha)
                    \quad \text{and} \quad
                    f \left( \bigcap_{\alpha \in A} E_\alpha \right) \subseteq
                    \bigcap_{\alpha \in A} f(E_\alpha)
                \]
            \item If $B$ and $C$ are subsets of $X$, then
                $f(C) \setminus f(B) \subseteq f(C \setminus B)$
            \item If ${\{E_\alpha\}}_{\alpha \in A}$ is a collection of subsets of $Y$, then
                \[
                    f^{-1} \left( \bigcup_{\alpha \in A} E_\alpha \right) =
                    \bigcup_{\alpha \in A} f^{-1} (E_\alpha)
                    \quad \text{and} \quad
                    f^{-1} \left( \bigcap_{\alpha \in A} E_\alpha \right) =
                    \bigcap_{\alpha \in A} f^{-1} (E_\alpha)
                \]
            \item If $B$ and $C$ are subsets of $Y$, then
                $f^{-1}(C \setminus B) = f^{-1}(C) \setminus f^{-1}(B)$.
            \item If $E \subseteq f(x)$, then $f(f^{-1}(E)) = E$, but if $E \subseteq X$,
                then $E \subseteq f^{-1}(f(E))$.
        \end{enumerate}

\end{itemize}

\subsection{Countable and Uncountable Sets}

\begin{itemize}
    \item \textbf{Definition 1.38} \emph{Countable & Uncountable}
        Let $E$ be a set.
        \begin{enumerate}
            \item $E$ is said to be \emph{finite} if and only if either $E = \emptyset$
                or there exists an injective function which takes
                $\{ 1, 2, \ldots, n \}$ onto $E$, for some $n \in \mathbb{N}$.
            \item $E$ is said to be \emph{countable} if and only if there exists and injective
                function which takes $\mathbb{N}$ onto $E$.
            \item $E$ is said to be \emph{at most countable} if and only if $E$ is either
                finite or countable.
            \item $E$ is said to be \emph{uncountable} if and only if $E$ is neither finite nor
                countable.
        \end{enumerate}

    \item \textbf{Remark 1.39} \emph{Cantor's Diagonalisation Argument} \\
        The open interval $(0, 1)$ is uncountable.

    \item \textbf{Lemma 1.40} \\
        A nonempty set $E$ is at most countable if and only if there is a function $g$ from
        $\mathbb{N}$ onto $E$.

    \item \textbf{Theorem 1.41} \\
        Suppose $A$ and $B$ are sets.
        \begin{enumerate}
            \item If $A \subseteq B$ and $B$ is at most countable, then $A$ is at most countable.
            \item If $A \subseteq B$ and $A$ is uncountable, then $B$ is uncountable.
            \item $\mathbb{R}$ is uncountable.
        \end{enumerate}

    \item \textbf{Theorem 1.42} \\
        Let $A_1, A_2, \ldots$ be at most countable sets.
        \begin{enumerate}
            \item Then $A_1 \times A_2$ is at most countable.
            \item If
                \[
                    E = \bigcup_{j=1}^\infty A_j :=
                    \bigcup_{j \in \mathbb{N}} A_j :=
                    \{ x : x \in A_j \ \text{for some} \ j \in \mathbb{N} \},
                \]
                then $E$ is at most countable.
        \end{enumerate}

    \item \textbf{Remark 1.43} \\
        The sets $\mathbb{Z}$ and $\mathbb{Q}$ are countable, but the set of irrationals is
        uncountable.

\end{itemize}

\break{}

\section{Sequences in $\mathbb{R}$}

\subsection{Limits of Sequences}

\begin{itemize}
    \item \textbf{Definition 2.1} \emph{Convergence} \\
        A sequence of real numbers $\{ x_n \}$ is set to \emph{converge} to a real number
        $a \in \mathbb{R}$ if and only if for every $\epsilon > 0$ there is an $N \in \mathbb{N}$
        (which in general depends on $\epsilon$) such that
        \[
            n \geq N \implies |x_n - a| < \epsilon
        \]

    \item \textbf{Remark 2.4}
        A sequence can have at most one limit.

    \item \textbf{Definition 2.5} \emph{Subsequence} \\
        By a \emph{subsequence} of a sequence ${\{x_n\}}_{n \in \mathbb{N}}$,
        we shall mean a sequence of the form ${\{x_n_k\}}_{k \in \mathbb{N}}$,
        where each $n_k \in \mathbb{N}$ and $n_1 < n_2 < \cdots$.

    \item \textbf{Remark 2.6} \\
        If ${ \{x_n\} }_{n \in \mathbb{N}}$ converges to $a$ and
        ${ \{x_n_k\} }{k \in \mathbb{N}}$ is any subsequence of
        ${ \{x_n\} }_{n \in \mathbb{N}}$, then $x_n_k$ converges to $a$ as $k \to \infty$.

    \item \textbf{Definition 2.7} \emph{Bounded Sequences} \\
        Let $\{ x_n \}$ be a sequence of real numbers.
        \begin{enumerate}
            \item The sequence $\{ x_n \}$ is said to be \emph{bounded above} if and only if
                the set $ \{ x_n : n \in \mathbb{N} \}$ is bounded above.
            \item The sequence $\{ x_n \}$ is said to be \emph{bounded below} if and only if
                the set $\{ x_n : n \in \mathbb{N} \}$ is bounded below.
            \item $\{ x_n \}$ is said to be \emph{bounded} if and only if it is bounded both
                above and below.
        \end{enumerate}

    \item \textbf{Theorem 2.8}
        Every convergent sequence is bounded.
\end{itemize}

\subsection{Limit Theorems}

\begin{itemize}
    \item \textbf{Theorem 2.9} \emph{Squeeze Theorem} \\
        Suppose that $\{x_n\}, \{y_n\}$, and $\{w_n\}$ are real sequences.
        \begin{enumerate}
            \item If $x_n \to a$ and $y_n \to a$ as $n \to \infty$,
                and if there is an $N_0 \in \mathbb{N}$ such that
                \[
                    x_n \leq w_n \leq y_n \ \text{for} \ n \geq N_0
                \]
                then $w_n \to a$ as $n \to \infty$.
            \item If $x_n \to 0$ as $n \to \infty$ and $\{y_n\}$ is bounded,
                then $x_n y_n \to 0$ as $n \to \infty$.
        \end{enumerate}

    \item \textbf{Theorem 2.11} \\
        Let $E \subset \mathbb{R}$.
        If $E$ has a finite supremum (respectively, a finite infimum),
        then there is a sequence $x_n \in E$ such that $x_n \to \sup E$
        (respectively, a sequence $y_n \in E$ such that $y_n \to \inf E$)
        as $n \to \infty$.

    \item \textbf{Theorem 2.12} \\
        Suppose that $\{x_n\}$ and $\{y_n\}$ are real sequences and that $\alpha \in \mathbb{R}$.
        If $\{x_n\}$ and $\{y_n\}$ are convergent, then
        \begin{enumerate}
            \item
                \[
                    \lim_{n \to \infty} (x_n + y_n) =
                    \lim_{n \to \infty} x_n + \lim_{n \to \infty} y_n
                \]
            \item
                \[
                    \lim_{n \to \infty} (\alpha x_n) = \alpha \lim_{n \to \infty} x_n
                \]
                and
            \item
                \[
                    \lim_{n \to \infty} (x_n y_n) =
                    ( \lim_{n \to \infty} x_n ) (\lim_{n \to \infty} y_n)
                \]
                If, in addition, $y_n \neq 0$ and $\lim_{n \to \infty} y_n \neq 0$, then
            \item
                \[
                    \lim_{n \to \infty} \frac{x_n}{y_n} =
                    \frac{\lim_{n \to \infty} x_n}{\lim_{n \to \infty} y_n}
                \]
                (In particular, all these limits exist.)

        \end{enumerate}

    \item \textbf{Definition 2.14} \emph{Divergence} \\
        Let $\{x_n\}$ be a sequence of real numbers.
        \begin{enumerate}
            \item $\{x_n\}$ is said to \emph{diverge} to $+\infty$ if and only if for each
                $M \in \mathbb{R}$ there is an $N \in \mathbb{N}$ such that
                \[
                    n \geq N \implies x_n > M
                \]
            \item $\{x_n\}$ is said to \emph{diverge} to $-\infty$ if and only if for each
                $M \in \mathbb{R}$ there is an $N \in \mathbb{N}$ such that
                \[
                    n \geq N \implies x_n < M
                \]
        \end{enumerate}

    \item \textbf{Theorem 2.15} \\
        Suppose that $\{x_n\}$ and $\{y_n\}$ are real sequences such that $x_n \to +\infty$
        (respectively, $x_n \to -\infty$) as $n \to \infty$.
        \begin{enumerate}
            \item If $y_n$ is bounded below (respectively, $y_n$ is bounded above), then
                \[
                    \lim_{n \to \infty} (x_n + y_n) = +\infty \quad
                    (\text{respectively}, \ \lim_{n \to \infty} (x_n + y_n) = -\infty)
                \]

            \item If $\alpha > 0$, then
                \[
                    \lim_{n \to \infty} (\alpha x_n) = +\infty \quad
                    (\text{respectively}, \ \lim_{n \to \infty} (\alpha x_n) = -\infty)
                \]

            \item If $y_n > M_0$ for some $M_0 > 0$ and all $n \in \mathbb{N}$, then
                \[
                    \lim_{n \to \infty} (x_n y_n) = +\infty \quad
                    (\text{respectively}, \ \lim_{n \to \infty} (x_n y_n) = -\infty)
                \]

            \item If $\{y_n\}$ is bounded and $x_n \neq 0$, then
                \[
                    \lim_{n \to \infty} \frac{y_n}{x_n} = 0
                \]

        \end{enumerate}

    \item \textbf{Corollary 2.16} \\
        Let $\{x_n\}$, $\{y_n\}$ be real sequences and $\alpha, x, y$ be extended real numbers.
        If $x_n \to x$ and $y_n \to y$, as $n \to \infty$, then
        \[
            \lim_{n \to \infty} (x_n + y_n) = x + y
        \]
        provided that the right side is not of the form $\infty - \infty$, and
        \[
            \lim_{n \to \infty} (\alpha x_n) = \alpha x, \quad \lim_{n \to \infty} (x_n y_n) = xy
        \]
        provided that none of these products is of the form $0 \cdot \pm \infty$.

    \item \textbf{Theorem 2.17} \emph{Comparison Theorem} \\
        Suppose that $\{x_n\}$ and $\{y_n\}$ are convergent sequences.
        If there is an $N_0 \in \mathbb{N}$ such that
        \[
            x_n \leq y_n \ \text{for} \ n \geq N_0
        \]
        then
        \[
            \lim_{n \to \infty} x_n \leq \lim_{n \to \infty} y_n
        \]
        In particular, if $x_n \in [a, b]$ converges to some point $c$,
        then $c$ must belong to $[a, b]$.

\end{itemize}

\subsection{Bolzano-Weierstrass Theorem}

\begin{itemize}
    \item \textbf{Definition 2.18} \emph{Increasing, Decreasing}
        Let ${\{x_n\}}_{n \in \mathbb{N}}$ be a sequence of real numbers.
        \begin{enumerate}
            \item $\{x_n\}$ is said to be \emph{increasing}
                (respectively, \emph{strictly increasing}) if and only if
                $x_1 \leq x_2 \leq \cdots (\text{respectively}, x_1 < x_2 < \cdots)$.
            \item $\{x_n\}$ is said to be \emph{decreasing}
                (respectively, \emph{strictly decreasing}) if and only if
                $x_1 \geq x_2 \geq \cdots (\text{respectively}, x_1 > x_2 > \cdots)$.
            \item $\{x_n\}$ is said to be \emph{monotone} if and only if it is either
                increasing or decreasing.
        \end{enumerate}

    \item \textbf{Theorem 2.19} \emph{Monotone Convergence Theorem} \\
        if $\{x_n\}$ is increasing and bounded above, or if $\{x_n\}$ is decreasing and bounded
        below, then $\{x_n\}$ converges to a finite limit.

    \item \textbf{Definition 2.22} \emph{Nested} \\
        A sequence of sets ${\{I_n\}}_{n \in \mathbb{N}}$ is said to be \emph{nested}
        if and only if
        \[
            I_1 \supseteq I_2 \supseteq \cdots
        \]


    \item \textbf{Theorem 2.23} \emph{Nested Interval Property} \\
        If ${\{I_n\}}_{n \in \mathbb{N}}$ is a nested sequence of nonempty closed bounded
        intervals, then $E := \bigcap_{n=1}^\infty I_n$ is nonempty.
        Moreover, if the lengths of these intervals satisfy $|I_n \to 0$ as $n \to \infty$
        then $E$ is a single point.

    \item \textbf{Remark 2.24}
        The Nested Interval Property might not hold if ``closed'' is omitted.

    \item \textbf{Remark 2.25}
        The Nested Interval Property might not hold if ``bounded'' is omitted.

    \item \textbf{Theorem 2.26} \emph{Bolzano-Weierstrass Theorem} \\
        Every bounded sequence of real numbers has a convergent subsequence.

\end{itemize}

\subsection{Cauchy Sequences}

\begin{itemize}
    \item \textbf{Definition 2.27} \emph{Cauchy} \\
        A sequence of points $x_n \in \mathbb{R}$ is said to be \emph{Cauchy} (in \mathbb{R})
        if and only if for every $\epsilon > 0$ there is an $N \in \mathbb{N}$ such that
        \[
            n, m \geq N \implies |x_n - x_m| < \epsilon
        \]

    \item \textbf{Remark 2.28}
        If $\{x_n\}$ is convergent, then $\{x_n\}$ is Cauchy.

    \item \textbf{Theorem 2.29} \emph{Cauchy} \\
        Let $\{x_n\}$ be a sequence of real numbers.
        Then $\{x_n\}$ is Cauchy if and only if $\{x_n\}$ converges
        (to some point $a \in \mathbb{R}$).

    \item \textbf{Remark 2.31}
        A sequence that satisfies $x_{n+1} - x_n \to 0$ is not necessarily Cauchy.

\end{itemize}

\subsection{Limits Supremum and Infimum}

\begin{itemize}
    \item \textbf{Definition 2.32} \emph{Limit Supremum \& Infimum} \\
        Let $\{x_n\}$ be a real sequence.
        Then the \emph{limit supremum} of $\{x_n\}$ is the extended real number
        \[
            \limsup_{n \to \infty} x_n := \lim_{n \to \infty} (\sup_{k \geq n} x_k)
        \]
        and the \emph{limit infimum} of $\{x_n\}$ is the extended real number
        \[
            \liminf_{n \to \infty} x_n := \lim_{n \to \infty} (\inf_{k \geq n} x_k)
        \]

    \item \textbf{Theorem 2.35} \\
        Let $\{x_n\}$ be a sequence of real numbers, $s = \limsup_{n \to \infty} x_n$,
        and $t = \liminf_{n \to \infty} x_n$.
        Then there are subsequences ${\{x_n_k\}}_{k \in \mathbb{N}}$ and
        ${\{x_\el_j\}}_{j \in \mathbb{N}}$ such that
        $x_n_k \to s$ as $k \to \infty$ and $x_\el_j \to t$ as $j \to \infty$.

    \item \textbf{Theorem 2.36} \\
        Let $\{x_n\}$ be a real sequence and $x$ be an extended real number.
        Then $x_n \to x$ as $n \to \infty$ if and only if
        \[
            \limsup_{n \to \infty} x_n = \liminf_{n \to \infty} x_n = x
        \]

    \item \textbf{Theorem 2.37} \\
        Let $\{x_n\}$ be a sequence of real numbers.
        Then $\limsup_{n \to \infty} x_n$ (respectively, $\liminf_{n \to \infty}$) is the largest
        value (respectively, the smallest value) to which some subsequences of $\{x_n\}$
        converges.
        Namely, if $x_n_k \to x$ as $k \to \infty$, then
        \[
            \liminf_{n \to \infty} x_n \leq x \leq \limsup_{n \to \infty} x_n
        \]

    \item \textbf{Remark 2.38}
        If $\{x_n\}$ is any sequence of real numbers, then
        \[
            \liminf_{n \to \infty} x_n \leq \limsup_{n \to \infty} x_n
        \]

    \item \textbf{Remark 2.39}
        A real sequence $\{x_n\}$ is bounded above if and only if
        $\limsup_{n \to \infty} x_n < \infty$, and is bounded below if and only if
        $\liminf_{n \to \infty} x_n > -\infty$.

    \item \textbf{Theorem 2.40} \\
        If $x_n \leq y_n$ for $n$ large, then
        \[
            \limsup_{n \to \infty} x_n \leq \limsup_{n \to \infty} y_n \quad \text{and} \quad
            \liminf_{n \to \infty} y_n \leq \liminf_{n \to \infty} y_n
        \]

\end{itemize}

\end{document}
